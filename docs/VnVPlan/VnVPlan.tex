\documentclass[12pt, titlepage]{article}
\usepackage[utf8]{inputenc}
\usepackage[margin=1in]{geometry}
\usepackage[titletoc,title]{appendix}
\usepackage{graphicx}
\usepackage{booktabs}
\usepackage{tabularx}
\usepackage{hyperref}
\usepackage{indentfirst}
\usepackage{fancyhdr}
\usepackage[dvipsnames]{xcolor}
\usepackage{pdflscape}
\usepackage{float}
\usepackage{multirow}
\usepackage{comment}
\hypersetup{ colorlinks, citecolor=blue, filecolor=black, linkcolor=red,
    urlcolor=blue }
\usepackage[round]{natbib}
\usepackage{soul}

\input{../Comments}
\input{../Common}

\begin{document}

\title{System Verification and Validation Plan for \progname{}} 
\author{\authname}
\date{\today}

	
\maketitle

\pagenumbering{roman}

\section{Revision History}

\begin{tabularx}{\textwidth}{p{3cm}p{2cm}X} \toprule {\bf Date} & {\bf Version}
& {\bf Notes}\\
\midrule
Nov. 2, 2022 & 1.0 & Revision 0\\
\bottomrule
\color{red}{Apr 2, 2023} & \color{red}{1.1} & \color{red}{Revision 1}\\
\bottomrule

\end{tabularx}

\newpage

\tableofcontents

\listoftables

\newpage

\section{Symbols, Abbreviations and Acronyms}

\begin{tabular}{p{0.4\linewidth}  p{0.5\linewidth}}
  \toprule		
  \textbf{Symbol/Abbreviation/Acronym} & \textbf{Description}\\
  \midrule 
  T & Test\\
  BE & Business Event\\
  SRS & Software Requirements Specification\\
  NFR & Non-Functional Requirement\\
  FR & Functional Requirement\\
  HTTP Request & A message sent by a client to initiate an action on the server
  through Hypertext Transfer Protocol\\
  \bottomrule
\end{tabular}\\

\newpage

\pagenumbering{arabic}

This document outlines caPstOneGroup's plan for verification and validation of
the correct and intended specification, implementation, and behaviour of our
software system, Park'd. It specifies plans to verify these elements, as well as
tests for the requirements in our
\href{https://github.com/parkd-app/park-d/blob/main/docs/SRS/SRS.pdf}{Software
Requirements Specification}.
\wss{provide an introductory blurb and roadmap of the Verification and Validation plan}

\section{General Information}
This section provides a brief overview of the software and its general
functions, gives the objective of the Verification and Validation plan, and
outlines relevant documentation to support the Verification and Validation plan.
\subsection{Summary}
We are testing the software named Park'd, our parking assistant application.
Park'd aims to help drivers find parking spaces by using machine learning
algorithms to locate empty spaces from overhead cameras. Our application then
directs drivers to those spaces, taking into account restrictions like reserved
or accessible spaces. It will maintain a database of spaces.
%as well as a navigation layout for a given parking lot.

\wss{Say what software is being tested.  Give its name and a brief overview of
  its general functions.}

\subsection{Objectives}
Our primary objective, through this Verification and Validation plan, is to
ensure that our product delivers on the functionality outlined in the functional
and non-functional requirements outlined in the
\href{https://github.com/parkd-app/park-d/blob/main/docs/SRS/SRS.pdf}{SRS}
without failures. Another objective of enacting this plan is to build confidence
in our software correctness and software consistency. By passing the tests
outlined in this report and thus, providing the intended outputs, we intend to
demonstrate correctness in our software's behaviour. It would also give a
measure of assurance in the consistency of our software, as the system should
ideally return the same output for a given input. We also wish to use this plan
to demonstrate a satisfactory level of usability, as passing the outlined tests
will give us the confidence that end users will not encounter unintended bugs or
glitches with the system.

\subsection{Relevant Documentation}

\wss{Reference relevant documentation.  This will definitely include your SRS
  and your other project documents (design documents, like MG, MIS, etc).  You
  can include these even before they are written, since by the time the project
  is done, they will be written.}

This Verification and Validation Plan document describes tests for functional
and non-functional requirements laid out in the
\href{https://github.com/parkd-app/park-d/blob/main/docs/SRS/SRS.pdf}{Software
Requirements Specification} document. Details of the components and modules
being tested can be found in the
\href{https://github.com/parkd-app/park-d/blob/main/docs/Design/MG/MG.pdf}{Module
Guide} and
\href{https://github.com/parkd-app/park-d/blob/main/docs/Design/MIS/MIS.pdf}{Module
Interface Specification} documents.

\section{Plan}
This section outlines plans to verify the SRS, the design, the Verification and
Validation plan itself, the implementation, and the automated testing.
Furthermore, it also outlines the plan to validate the software. 

\wss{Introduce this section.   You can provide a roadmap of the sections to
  come.}

\subsection{Verification and Validation Team}
\begin{table}[hp]
\centering
\begin{tabular}{|c|c|c|}
\hline
\textbf{Name} & \textbf{Roles} & \textbf{Documentation}\\
\hline
Albert & Implementation Verification &
\hyperref[sec:implementation]{Implementation}\\
Almen & Automated Testing and Verification Tools &
\hyperref[sec:automation]{Automation}\\
David & SRS Verification & \hyperref[sec:srs]{SRS}\\
Gary & Software Validation & \hyperref[sec:validation]{Validation}\\
Jonathan & Verification and Validation Verification & \hyperref[sec:vnv]{VnV}\\
Kabishan & Design Verification & \hyperref[sec:design]{Design}\\
\hline
\end{tabular}
\caption{Team Members}
\end{table}

Each individual is in charge of the plan whose role they are assigned. They are
not solely responsible for the implementation of the relevant plan.
\textcolor{red}{Any team member who has finished their testing can support
others, particularly those whose testing is the most time consuming or who has
the least \% of testing completed.} \wss{Your teammates.  Maybe your supervisor.
You shoud do more than list names.  You should say what each person's role is
for the project's verification.  A table is a good way to summarize this
information.}

\subsection{SRS Verification Plan}
\label{sec:srs}
The
\href{https://github.com/parkd-app/park-d/blob/main/docs/SRS/SRS.pdf}{Software
Requirements Specification} verification plan ensures that our requirements
efficiently cover the breadth of the system, and that they are feasible to
implement. Verification will be conducted with group members, as well as
classmates. It will be in the form of a questionnaire to be applied to some or
all requirements, as well as ad hoc feedback where necessary. Given the large
number of requirements in our project, an individual should not necessarily
review every requirement.

Our SRS comprises project drivers which include the purpose and the
stakeholders, project constraints, functional and non-functional requirements,
and potential project issues. Each component should be covered by the
questionnaire, and reviewed thoroughly by reviewers.

The questionnaire should cover the following points:
\begin{itemize}
    \item What stakeholders are missing? Are there any listed that are
    irrelevant?
    \item Are there any missed constraints?
    \item Are there any erroneous facts or assumptions?
    \item Do requirements specify "what", and not "how"?
    \item Are requirements unambiguous, verifiable, and abstract?
    \item What requirements are missing? Are there any listed that are
    unnecessary?
    \item Are there any potential issues that are not listed?
\end{itemize}

The SRS verification plan aims to cover the entire document, and is considered
complete when there are no outstanding issues.

\wss{List any approaches you intend to use for SRS verification.  This may
  include ad hoc feedback from reviewers, like your classmates, or you may plan
  for something more rigorous/systematic.}

\wss{Maybe create an SRS checklist?}

% Kabishan's part

\subsection{Design Verification Plan}
\label{sec:design}
The design verification plan will build confidence that our design
infrastructure will take the necessary inputs and deliver the intended outputs
with completeness and correctness. 

Our system is multi-tiered, in that it is comprised of a back-end and front-end
system. The back-end system consists of a machine learning model that will
identify the parking spaces and their availability, and path finding algorithms
that direct the driver to their desired parking space. The front-end system
allows the user to interact with this information. Therefore, the design
verification plan will be twofold. The design verification for the back-end
system will be performed by our internal team members, as we possess intimate
knowledge of the frameworks, and thus, would be able to identify the subtleties
as to where our design infrastructure could fail. 

For the back-end system, we will be verifying the following:
\begin{itemize}
    \item Machine learning model accepts the camera and sensor footage
    \item Machine learning model identifies normal, accessibility, and reserved
    parking spaces from the camera and sensor footage
    \item Machine learning model updates the availability of a parking space
    based on the presence of a vehicle and the driver's geographical information
    \item Machine learning model outputs the aforementioned information to the
    front-end system
    \item Path finding algorithm accepts the driver's geographical information
    and their desired parking space
    \item Path finding algorithm calculates instructions to navigate the driver
    to their desired parking space
    \item Path finding algorithm delivers these instructions to the front-end
    system
\end{itemize}

The front-end system consists of an interface, authored in a framework of our
choice that will communicate with the back-end to retrieve parking availability
information and driving instructions. It will also send requests to the back-end
system to update information, such as when a driver selects a parking space, at
which point, the spot becomes unavailable. For the outward facing interface, our
design verification approach involves both internal team members and external
members, such as fellow classmates. This approach will help us eliminate any
biases that we, the developers, may have against testing the system and this
allows others to try to break the system.

For the front-end system, we will be verifying the following:
\begin{itemize}
    \item Interface retrieves normal, reserved, and accessibility parking space
    information from back-end system
    \item Interface displays the aforementioned information to the user
    \item Interface accepts input from the driver to change an available parking
    space to unavailable
    \item Upon selecting a parking space, the interface provides the back-end
    with the driver's geographical information
    \item Interface will accept the driving instructions from the back-end
    system
    \item Interface will display the driving instructions to the driver
\end{itemize}

\subsection{Verification and Validation Plan Verification Plan}
\label{sec:vnv}

\wss{The verification and validation plan is an artifact that should also be verified.}

\wss{The review will include reviews by your classmates}

\wss{Create a checklists?}

This Verification and Validation Plan also needs to be verified and validated.
The plan to accomplish this will include reviews and checklists to ensure that
every aspect of the Verification and Validation plan accurately and sufficiently
describe the development team's plans to verify and validate the system. A
review will be completed by the the members of the development team as well as
our peers. This review will cover the following points:
\begin{itemize}
    \item Plan to verify and validate the Software Requirements Specification of
    the system is described and explained in a clear manner
    \item Plan to verify and validate the design of the system is described and
    explained in a clear manner
    \item Plan to verify and validate the implementation of the system is
    described and explained in a clear manner
    \item Tools that will be used for automated testing and verification are
    described and explained in a clear manner
    \item Plan to verify and validate the software of the system is described
    and explained in a clear manner
    \item Tests for all business events are explained thoroughly are deemed
    valid tests for its requirement
\end{itemize}

After reviewing our peers' feedback, if all members of the development team
agree that the Verification and Validation Plan satisfies the above criteria,
the plan is considered complete.

\subsection{Implementation Verification Plan}
\label{sec:implementation}

\wss{You should at least point to the tests listed in this document and the unit
  testing plan.}

\wss{In this section you would also give any details of any plans for static
  verification of the implementation.  Potential techniques include code
  walkthroughs, code inspection, static analyzers, etc.}
  
The implementation verification plan will verify that our implementation has the
correct behaviour specified in our
\href{https://github.com/parkd-app/park-d/blob/main/docs/SRS/SRS.pdf}{SRS}. The
plan will be split in two: \hyperref[sec:5.1]{functional} and
\hyperref[sec:5.2]{nonfunctional} requirements.

For functional requirements, we will split them up according to the back end and
front end system. For the back end, testing the machine learning algorithm will
be done by dynamic, manual, and automatic testing. At first, we will test
manually with a small set of data. Then we will test automatically with a large
amount of data to develop the AI further. For the requirements that use the
machine learning algorithm, we will use manual dynamic tests and verify the
correct output. For the front end, we will use automatic dynamic tests for
requirements such as logging in. The remaining will be done manually.

For non-functional requirements, we will split them up based on system and user
requirements. System requirements will be verified with dynamic testing, similar
to functional requirements. Instead of checking the output we will verify the
system's properties. For \hyperref[sec:5.2.2]{user requirements}, we will create
a \hyperref[sec:7.2]{usability survey} and evaluate based on user feedback.

\subsection{Automated Testing and Verification Tools}
\label{sec:automation}

\wss{What tools are you using for automated testing.  Likely a unit testing
  framework and maybe a profiling tool, like ValGrind.  Other possible tools
  include a static analyzer, make, continuous integration tools, test coverage
  tools, etc.  Explain your plans for summarizing code coverage metrics. Linters
  are another important class of tools.  For the programming language you
  select, you should look at the available linters.  There may also be tools
  that verify that coding standards have been respected, like flake9 for
  Python.}

\wss{If you have already done this in the development plan, you can point to
that document.}

\wss{The details of this section will likely evolve as you get closer to the
  implementation.}
  
The automation testing and verification tools is outlined and discussed in the
\href{https://github.com/parkd-app/park-d/blob/main/docs/DevelopmentPlan/DevelopmentPlan.pdf}{Development
Plan}.

With regards to code coverage metrics, as mentioned in the
\href{https://github.com/parkd-app/park-d/blob/main/docs/DevelopmentPlan/DevelopmentPlan.pdf}{Development
Plan}, the plan is to use
\href{https://coverage.readthedocs.io/en/6.5.0/}{Coverage.py} and
\href{https://istanbul.js.org/}{Istanbul} to measure the code coverage for our
Python and Cypress test cases respectively. The goal is to achieve a minimum of
80\% code coverage, putting emphasis on creating high quality test cases that
test the most critical components of the system.

\subsection{Software Validation Plan}
\label{sec:validation}

To address the problem of developers including features based on their own
understanding and changing requirements without effectively communicating with
the client, a software validation plan is needed to determine whether the
software satisfies specified requirements. We intend to hold a review session
with the stakeholders to verify whether we have built the right product.
Different review sessions should be held with different types of stakeholders.
For example, functional requirements 1 to 25 are verified in collaboration with
a driver who uses the software, whereas the rest is verified with the parking
lot owner/manager.

The plan is as follows: 
\begin{itemize}
    \item Each requirement is tested by conducting an acceptance test with the
    stakeholders. 
    \item The acceptance test scenarios and test cases are identified from the
    business events from the SRS document. The test cases should cover all the
    business events.
    \item Execute test cases and report bugs if any. Test results should be
    recorded accordingly and re-test bugs once fixed.
    \item Confirm with the stakeholders whether the business objective has been
    met. The team must resolve any issue discovered in the previous steps before
    this.
\end{itemize}


\section{System Test Description}
	
\subsection{Tests for Functional Requirements}
\label{sec:5.1}
Every functional requirement in the
\href{https://github.com/parkd-app/park-d/blob/main/docs/SRS/SRS.pdf}{SRS} is
assigned a corresponding test in this section. Non-functional requirements are
tested in the next section.

\wss{Include a blurb here to explain why the sections below cover the
  requirements. References to the SRS would be good here. If a section covers
  tests for input constraints, you should reference the data constraints table
  in the SRS.}
		
\subsubsection{BE1}

\begin{enumerate}

\item{BE1-FR1-T1}

Control: Functional, Dynamic, Manual
					
Initial State: System started, device location is disabled
					
Input: N/A
					
Output: System prompts the driver to turn on location.

Test Case Derivation: Location needs to be enabled to allow the application to
direct the user to a space.
					
How test will be performed: Tester will launch the application while their
device's location is disabled.
					
\item{BE1-FR2-T1}

Control: Functional, Dynamic, Manual
					
Initial State: System started, device location is enabled
					
Input: N/A
					
Output: System displays to the driver a map of their surroundings.

Test Case Derivation: The driver needs to see where they are located relative to
nearby parking lots so they can navigate to the desired lot.

How test will be performed: Tester will launch the application with their
device's location enabled.

\textcolor{red}{\item{BE1-FR3-T1}}

\textcolor{red}{Control: Functional, Dynamic, Manual}

\textcolor{red}{Initial State: System started, device location is disabled}
					
\textcolor{red}{Input: N/A}
					
\textcolor{red}{Output: System displays to the driver a navigation disabled message.}

\textcolor{red}{Test Case Derivation: The driver needs to know that the system won't show their location or nearby parking lots.}

\textcolor{red}{How test will be performed: Tester will launch the application
with their device's location disabled.}

\end{enumerate}

\subsubsection{BE2}

\begin{enumerate}

\item{BE2-FR4-T1}

Control: Structural, Dynamic, Automatic
					
Initial State: The backend machine vision system is running
					
Input: Video feed
					
Output: The backend system recognizes the video feed and processes the image.

Test Case Derivation: A video feed will need to be fed to the system for it to
do its work. The function of recognizing parking spaces is meant to be fulfilled
using this feed.
					
How test will be performed: The backend system will be run with a given video.

\item{BE2-FR5-T1}

Control: Structural, Dynamic, Automatic
					
Initial State: The backend system is running with a video feed displaying a
disabled space.
					
Input: Test space that is known to be marked as disabled
					
Output: The backend system recognizes the space and marks it in the database as
a disabled space.

Test Case Derivation: Disabled spaces are to be marked accordingly so they are
not suggested to drivers who cannot use them.
					
How test will be performed: The system will run with a known disabled space to
ascertain correct functioning of its space detection.

\item{BE2-FR6-T1}

Control: Structural, Dynamic, Automatic
					
Initial State: The backend system is running with a video feed displaying a
valid space
					
Input: Test space with known location
					
Output: The test space and the correct location input to the database.

Test Case Derivation: The location given by the system should be the same as
that determined by us beforehand \textcolor{red}{with an accuracy of at least
$\hyperlink{distance_tolerance}{DISTANCE\_TOLERANCE}$ meters.}.
					
How test will be performed: The system will run with a space with known location
to ascertain correct functioning of its space detection.

\item{BE2-FR7-T1}

Control: Functional, Dynamic, Manual
					
Initial State: System running with driver location enabled
					
Input: Driver selects the option to navigate to a parking lot
					
Output: System displays a prompt for the driver to select their desired parking
area.

Test Case Derivation: The system should make it obvious that the driver should
select a destination.
					
How test will be performed: Tester will select the option to navigate to a
parking lot with location enabled.

\item{BE2-FR8-T1}

Control: Functional, Dynamic, Manual
					
Initial State: System running with driver location enabled, parking lot selected
					
Input: N/A
					
Output: System displays the number of spaces available at the selected
destination.

Test Case Derivation: The system should inform the driver of conditions at their
selected destination before they arrive.
					
How test will be performed: Tester will select the option to navigate to a
parking lot with location enabled and observe the output.

\end{enumerate}

\subsubsection{BE3}

\begin{enumerate}

\item{BE3-FR9-T1}

Control: Functional, Dynamic, Manual
					
Initial State: System running with driver location enabled, parking lot selected
					
Input: Driver selects the option to filter the visible spaces
					
Output: Depending on the input, only normal, accessible, and reserved spaces are
displayed.

Test Case Derivation: Some drivers need to be able to find specially designated
spaces.
					
How test will be performed: Tester will select the option to filter the visible
spaces at a selected parking lot.

\item{BE3-FR10-T1}

Control: Functional, Dynamic, Manual
					
Initial State: System running with driver location enabled, parking lot selected
					
Input: N/A
					
Output: Reserved and disabled spaces are not displayed.

Test Case Derivation: These spaces are not to be displayed to drivers unless
they are qualified to use them.
					
How test will be performed: Tester will observe the output of the system with a
selected parking lot and no further inputs.

\end{enumerate}

\subsubsection{BE4}

\begin{enumerate}

\item{BE4-FR11-T1}

Control: Functional, Dynamic, Manual
					
Initial State: System running with driver location enabled, parking lot selected
					
Input: Driver selects a parking space at the selected parking lot
					
Output: System displays the selected parking space.

Test Case Derivation: The system should give feedback that the driver has
selected a space, and that it will work with that space.
					
How test will be performed: Tester will select a parking space.

\item{BE4-FR12-T1}

Control: Functional, Dynamic, Manual
					
Initial State: System started, parking spaces available, but no parking space
selected
					
Input: Selecting the parking space
					
Output: Driver is shown directions from their location to the desired parking
space.

Test Case Derivation: The directions should begin at the driver's location and
end at the parking space's location, otherwise, the driver will be provided
wrong directions.
					
How test will be performed: Tester will pick a parking space and validate that
the directions take them from their current position to the parking space.

\item{BE4-FR13-T1}

Control: Functional, Dynamic, Manual
					
Initial State: System started, parking spaces available, but no parking space
selected
					
Input: Driver selects default recommendation for parking space
					
Output: Driver is given directions to reach the parking space provided as a
recommendation.

Test Case Derivation: To limit the driver's interaction with the interface
whilst driving, the recommendation should be available for parking and should
meet the needs of the user (e.g. default accessibility parking for those with
disabilities)
					
How test will be performed: Tester will toggle between normal and accessibility
parking modes and check that the system provides a default parking space that is
available.

\item{BE4-FR14-T1}

Control: Functional, Dynamic, Manual
					
Initial State: System started, parking spaces available, but no parking space
selected
					
Input: N/A
					
Output: The system will display navigation directions to the default recommended
parking space after $\hyperlink{default_delay}{DEFAULT\_DELAY}$ seconds.

Test Case Derivation: After $\hyperlink{default_delay}{DEFAULT\_DELAY}$ seconds,
the driver must receive instructions, since they should minimize their
interaction with the interface while driving.
					
How test will be performed: Tester will toggle between normal and accessibility
parking modes and wait for $\hyperlink{default_delay}{DEFAULT\_DELAY}$ seconds
to check that the system provides directions to a default parking space that is
available.

\end{enumerate}

\subsubsection{BE5}

\begin{enumerate}

\item{BE5-FR15-T1}

Control: Functional, Dynamic, Manual
					
Initial State: System started 
					
Input: Driver selects a parking space that is unavailable.
					
Output: System informs the driver that they cannot park in a spot that is
already occupied by another vehicle.

Test Case Derivation: Driver should not be allowed to park in a spot that houses
another vehicle.
					
How test will be performed: Tester will attempt to select an unavailable parking
space and validate that they receive a notification from the system.

\item{BE5-FR16-T1}

Control: Functional, Dynamic, Manual
					
Initial State: System started, reserved or accessibility parking spaces
available, driver has not filtered out reserved or accessibility spaces
					
Input: Driver selects a reserved or accessibility parking space.
					
Output: The system will inform the driver that they have tried to select a
parking space that does not meet their needs. 

Test Case Derivation: Drivers should not be allowed to park in spaces that they
are not entitled to.
					
How test will be performed: Tester will not filter reserved and accessibility
parking spaces. Upon selecting one of the aforementioned spaces, the tester will
be notified by the system.

\end{enumerate}

\subsubsection{BE6}

\begin{enumerate}

\item{BE6-FR17-T1}

Control: Functional, Dynamic, Manual
					
Initial State: System started, parking space selected
					
Input: Driver's geographical location
					
Output: As the driver commutes, the directions should update such that the
parking space remains stationary and the user's location is dynamic

Test Case Derivation: If there are obstacles on the road, the driver's change in
location should lead the system to recalculate the path to the parking space
					
How test will be performed: Tester will follow the instructions and ask another
tester to stand in front of their vehicle. The directions should change at this
moment.

\item{BE6-FR18-T1}

Control: Functional, Dynamic, Manual
					
Initial State: System started, parking space selected, other parking spaces
available
					
Input: N/A
					
Output: Once the initial parking space is no longer available, the system shows
a recommendation to the driver.

Test Case Derivation: Driver should not be led to a parking space that is taken,
as this will result in frustration and traffic congestion.
					
How test will be performed: Tester will drive to the initial parking space.
Another tester will park at the initial parking space. The first tester will
receive a recommendation to another parking space.

\item{BE6-FR19-T1}

Control: Functional, Dynamic, Manual
					
Initial State: System started, parking space selected, other parking spaces
available
					
Input: N/A
					
Output: Once the initial parking space is no longer available, the system lets
the user pick another space from those that are available.

Test Case Derivation: User should be allowed to pick another parking space if
their initial parking space is taken.
					
How test will be performed: Tester will drive to the initial parking space.
Another tester will park at the initial parking space. The first tester will be
given the ability to pick another parking space.

\textcolor{red}{\item{BE6-FR20-T1}}

\textcolor{red}{Control: Functional, Dynamic, Manual}

\textcolor{red}{Initial State: System started, parking space selected, parking space taken}
					
\textcolor{red}{Input: N/A}
					
\textcolor{red}{Output: System terminates the route and informs the driver to select and new spot.}

\textcolor{red}{Test Case Derivation: The driver needs to know if the spot they're going to is taken by someone else.}

\textcolor{red}{How test will be performed: Tester will drive to the initial
parking space. Another tester will park at the initial parking space. The first
tester will be informed of the change.}

\textcolor{red}{\item{BE6-FR21-T1}}

\textcolor{red}{Control: Functional, Dynamic, Manual}

\textcolor{red}{Initial State: System started, parking space selected}
					
\textcolor{red}{Input: N/A}
					
\textcolor{red}{Output: System terminates the route after a time estimate plus
$\hyperlink{timeout_tolerance}{TIMEOUT\_TOLERANCE}$ seconds and informs the
driver to select and new spot.}

\textcolor{red}{Test Case Derivation: The driver needs to know if they've taken too long to drive to the spot}

\textcolor{red}{How test will be performed: Tester will take a long drive to the parking space. The  tester will be informed of the timeout}

\textcolor{red}{\item{BE6-FR22-T1}}

\textcolor{red}{Control: Functional, Dynamic, Manual}

\textcolor{red}{Initial State: System started}
					
\textcolor{red}{Input: N/A}
					
\textcolor{red}{Output: System displays current location within
$\hyperlink{distance_tolerance}{DISTANCE\_TOLERANCE}$ meters}

\textcolor{red}{Test Case Derivation: The driver should have an accurate estimation of their location on the map.}

\textcolor{red}{How test will be performed: Tester will start the system and note the location on the map compared to their known location.}

\textcolor{red}{\item{BE6-FR23-T1}}

\textcolor{red}{Control: Functional, Dynamic, Manual}

\textcolor{red}{Initial State: System started, parking spot selected}
					
\textcolor{red}{Input: N/A}
					
\textcolor{red}{Output: System displays a time estimate to the destination}

\textcolor{red}{Test Case Derivation: The driver should have an accurate
estimation of the driving time within
$\hyperlink{driving_estimate_tolerance}{DRIVING\_ESTIMATE\_TOLERANCE}$ seconds.}

\textcolor{red}{How test will be performed: Tester will drive to a selected parking spot and compare the total driving time with the initial estimate.}

\end{enumerate}

\subsubsection{BE7}

\begin{enumerate}

\item{BE7-FR24-T1}

Control: Functional, Dynamic, Manual
					
Initial State: System started, parking spot selected
					
Input: Driver's geographical location
					
Output: The system shall inform the driver when they have reached their parking
spot.

Test Case Derivation: If the driver is not informed, they might not be aware
that they have reached the parking spot of their selection.
					
How test will be performed: Tester will drive to their selected parking space
and validate that they receive a notification.

\item{BE7-FR25-T1}

Control: Functional, Dynamic, Manual
					
Initial State: System started
					
Input: Driver's geographical location
					
Output: The parking space changes from available to unavailable.

Test Case Derivation: Other drivers are aware that the parking space is
unavailable, such that they will not pick the location when parking in the
future.
					
How test will be performed: Tester confirms that while driving, the selected
parking space is available. When they park, the selected parking space should
become unavailable.

\item{BE7-FR26-T1}

Control: Functional, Dynamic, Manual
					
Initial State: System started, parking spot reached
					
Input: Driver's feedback
					
Output: The feedback is saved.

Test Case Derivation: Storing the driver's feedback will allow the team to
detect issues and dissatisfaction with the system.
					
How test will be performed: Tester will reach a parking spot, verify that they
receive a feedback prompt, and confirm the feedback is saved.

\end{enumerate}

\subsubsection{BE8}

\begin{enumerate}
    
    \item{\textcolor{red}{\st{BE8-FR27-T1}}}

    \textcolor{red}{\st{Control: Functional, Dynamic, Manual}}
    					
    \textcolor{red}{\st{Initial State: System started}}
    					
    \textcolor{red}{\st{Input: New settings}}
    					
    \textcolor{red}{\st{Output: The system settings change to the ones given}}
    
    \textcolor{red}{\st{Test Case Derivation: Not changing the settings would
    make being able to enter new ones useless.}}
    					
    \textcolor{red}{\st{How test will be performed: Tester will check the
    current settings, apply new ones, and verify the change to the system.}}

    \item{\textcolor{red}{\st{BE8-FR28-T1}}}

    \textcolor{red}{\st{Control: Functional, Dynamic, Manual}}
    					
    \textcolor{red}{\st{Initial State: System started}}
    					
    \textcolor{red}{\st{Input: N/A}}
    					
    \textcolor{red}{\st{Output: System settings including volume and sound
    adjustment, unit identification, vehicle details, and notification
    preferences}}
    
    \textcolor{red}{\st{Test Case Derivation: These settings are specified under
    the functional requirement.}}
    					
    \textcolor{red}{\st{How test will be performed: Tester will open the
    settings and verify these settings are present.}}
\end{enumerate}

\subsubsection{BE9}

\begin{enumerate}

    \item{BE9-FR29-T1}

    Control: Functional, Dynamic, Manual
    					
    Initial State: System started, parking spot selected, parking spot path
    found
    					
    Input: Cancel signal
    					
    Output: Parking spot selection and path are discarded
    
    Test Case Derivation: The driver is no longer going to this spot so they
    should no longer be given the path.
    					
    How test will be performed: Tester will choose a parking slot to go to,
    cancel the path, and verify that the selection and path are gone.

    \item{BE9-FR30-T1}

    Control: Functional, Dynamic, Manual
    					
    Initial State: System started, parking spot selected, parking spot path
    found
    					
    Input: Cancel signal
    					
    Output: The driver's location
    
    Test Case Derivation: Once the path is terminated, the driver will may to
    choose another parking spot.
    					
    How test will be performed: Tester will choose a parking slot to go to,
    cancel the path, and verify that they return to spot selection.

\end{enumerate}

\subsubsection{BE10}

\begin{enumerate}

    \item{BE10-FR31-T1}

    Control: Functional, Dynamic, Manual
    					
    Initial State: System started
    					
    Input: N/A
    					
    Output: Admin console login
    
    Test Case Derivation: The admin console is a protected and is intended only
    for parking lot owners/managers.
    					
    How test will be performed: Tester will verify the admin console requires
    admin credentials.

    \item{BE10-FR32-T1}

    Control: Functional, Dynamic, Manual
    					
    Initial State: System started
    					
    Input: Admin credentials
    					
    Output: Admin console
    
    Test Case Derivation: The admin console is a protected console for parking
    lot owners/managers.
    					
    How test will be performed: Tester will verify the admin console is
    accessible only with the admin credentials.

    \item{BE10-FR33-T1}

    Control: Functional, Dynamic, Manual
    					
    Initial State: System started, accessing admin console
    					
    Input: N/A
    					
    Output: Occupancy map
    
    Test Case Derivation: The occupancy map will show the availability of all
    parking spaces in the parking lot as specified under the functional
    requirement.
    					
    How test will be performed: Tester will access the admin console and verify
    the occupancy map is accurate to the parking lot.
\end{enumerate}

\subsubsection{BE11}
\begin{enumerate}
    \item{BE11-FR34-T1}

    Control: Functional, Dynamic, Manual
    					
    Initial State: System started, accessing admin console
    					
    Input: N/A
    					
    Output: Parking analytics
    
    Test Case Derivation: The parking analytics will show live data of the
    availability of all parking spaces in the parking lot as specified under the
    functional requirement.
    					
    How test will be performed: Tester will access the admin console and verify
    the live parking analytics are given.

    \textcolor{red}{\item{BE11-FR35-T1}}
    
    \textcolor{red}{Control: Functional, Dynamic, Manual}
    
    \textcolor{red}{Initial State: System started, parking lot selected}
    					
    \textcolor{red}{Input: N/A}
    					
    \textcolor{red}{Output: Graph with occupancy levels}
    
    \textcolor{red}{Test Case Derivation: Twenty-four hours worth of historical information is required to allow the user/admin to make an informed decision about choosing a particular spot, or conducting maintenance work.}
    
    \textcolor{red}{How test will be performed: Tester will see if live data for a particular hour is within 20\% of the historical data for that hour.}
\end{enumerate}

\subsubsection{BE12}

\begin{enumerate}
    \item{BE12-FR36-T1}

    Control: Functional, Dynamic, Manual
    					
    Initial State: System started, accessing admin console
    					
    Input: Parking lot layout change
    					
    Output: The parking lot layout changes to the one given
    
    Test Case Derivation: Parking lot owners/managers will want to modify the
    layout in case it changes.
    					
    How test will be performed: Tester will access the admin console, make a
    change to the parking lot layout, and verify the new layout is kept.

    \item{BE12-FR37-T1}

    Control: Functional, Dynamic, Manual
    					
    Initial State: System started, accessing admin console
    					
    Input: Parking lot spot change
    					
    Output: The parking lot spot changes to the one given
    
    Test Case Derivation: Parking lot owners/managers will want to redefine a
    parking spot in case of adding/removing/moving reserved spots.
    					
    How test will be performed: Tester will access the admin console, redefine a
    parking spot, and verify the spot has changed.

    \end{enumerate}
    
\subsection{Tests for Nonfunctional Requirements}
\label{sec:5.2}

\wss{The nonfunctional requirements for accuracy will likely just reference the
  appropriate functional tests from above.  The test cases should mention
  reporting the relative error for these tests.  Not all projects will
  necessarily have nonfunctional requirements related to accuracy}

\wss{Tests related to usability could include conducting a usability test and
  survey. The survey will be in the Appendix.}

\wss{Static tests, review, inspections, and walkthroughs, will not follow the
format for the tests given below.}

Non functional requirements can be found in the
\href{https://github.com/parkd-app/park-d/blob/main/docs/SRS/SRS.pdf}{SRS}. They
are organized by type and subtype.

\subsubsection{Look and Feel Requirements}
\label{sec:5.2.1}
\paragraph{Appearance Requirements}

\begin{enumerate}

\item{NFR-LF1-T1}

Type: Functional, Dynamic, Manual
					
Initial State: System is launched
					
Input/Condition: Go through all pages and uses all interface functions and
features
					
Output/Result: N/A
					
How test will be performed: Each tester will go through the whole app and try to
use all available functions and features. If an interface feature, such as a
button, is unanimously deemed unnecessary, it will be removed.
					
\end{enumerate}

\paragraph{Style Requirements}

\begin{enumerate}

\item{NFR-LF2-T1}

Type: Functional, Dynamic, Manual
					
Initial State: System is launched
					
Input/Condition: Go through all pages and uses all interface functions and
features
					
Output/Result: N/A

How test will be performed: Navigation app users will be surveyed. Users will be
given $\hyperlink{exploration_time}{EXPLORATION\_TIME}$ to explore the app,
after which they will answer whether or not the app is visually/stylistically
similar to existing popular navigation apps such as Google Maps, Apple Maps, and
Waze. \color{red}{At least 5 users will be surveyed. At least 80\% of users
surveyed should indicate that there is a similarity.}
					
\end{enumerate}

\subsubsection{Usability and Humanity Requirements}
\label{sec:5.2.2}
\paragraph{Ease of Use Requirements}

\begin{enumerate}

\item{NFR-UH1-T1}

Type: Functional, Dynamic, Manual
					
Initial State: System is launched
					
Input/Condition: Go through all pages and uses all interface functions and
features
					
Output/Result: N/A
					
How test will be performed: Testers will test all functions considered atomic
and make sure that none take more than $\hyperlink{max_taps}{MAX\_TAPS}$,
defined in the SRS document, to complete.
				
\item{NFR-UH2-T1}

Type: Functional, Dynamic, Manual
					
Initial State: System is launched
					
Input/Condition: Go through all pages and uses all interface functions and
features
					
Output/Result: N/A
					
How test will be performed: Navigation app users will be surveyed. Users will be
given $\hyperlink{exploration_time}{EXPLORATION\_TIME}$ to explore the app,
after which they will answer whether or not the app is as easy to use as to
existing popular navigation apps such as Google Maps, Apple Maps, and Waze.
\color{red}{At least 5 users will be surveyed. At least 80\% of users surveyed
should indicate that it is as easy to use.}
					
\end{enumerate}

\paragraph{Learning Requirements}

\begin{enumerate}

\item{NFR-UH3-T1}

Type: Functional, Dynamic, Manual
					
Initial State: System is launched
					
Input/Condition: Go through steps to find an available parking spot
					
Output/Result: N/A
					
How test will be performed: Random variety of users will be surveyed. Users will
be given $\hyperlink{exploration_time}{EXPLORATION\_TIME}$ to figure out how to
find and navigate to an available parking spot without any prior instructions.
Testers will record how many users successfully complete the task.
\color{red}{At least 5 users will be surveyed. At least 80\% of users should
successfully complete the task.}
					
\end{enumerate}

\paragraph{Understandability and Politeness Requirements}

\begin{enumerate}

\item{NFR-UH4-T1}

Type: Functional, Dynamic, Manual
					
Initial State: System is launched
					
Input/Condition: Valid address of a parking lot
					
Output/Result: Symbols identifying special parking spaces (i.e. accessibility,
electric vehicle, or reserved) are displayed.
					
How test will be performed: Testers will check that all graphic files of symbols
used in the application conform to any regulations set out by the Ontario
Ministry of Transportation.
					
\end{enumerate}

\subsubsection{Performance Requirements}
\label{sec:5.2.3}
\paragraph{Speed and Latency Requirements}

\begin{enumerate}

\item{NFR-PE1-T1}

Type: Functional, Dynamic, Manual
					
Initial State: System is launched
					
Input/Condition: Use all the functions and features of the system
					
Output/Result: The system must maintain a minimum of
$\hyperlink{min_framerate}{MIN\_FRAMERATE}$ at all times
					
How test will be performed: This test would be performed on multiple devices
(i.e. PC, mobile device). The tester would open up the system on a device and
monitor the framerate while performing operations on the application.
					
\end{enumerate}

\paragraph{Safety-Critical Requirements}

\begin{enumerate}

\item{NFR-PE2-T1}

Type: Functional, Dynamic, Manual
					
Initial State: System is launched in a vehicle
					
Input/Condition: Address of a parking lot is entered and an empty parking spot
located near a sidewalk or at the corner of a parking lot is selected.
					
Output/Result: The system successfully provides directions without going on a
sidewalk or off the road.
					
How test will be performed: The tester will launch the system on their device
and select an available parking spot located beside a sidewalk. The tester will
then follow the directions provided by the system until the tester reaches the
designated parking spot without going off the road or on a sidewalk.

\item{NFR-PE3-T1}

Type: Functional, Dynamic, Manual
					
Initial State: System is launched in a vehicle
					
Input/Condition: Address of a parking lot is entered and an empty parking spot
located near a restricted area like a fire lane is selected.
					
Output/Result: System directs the user to park only in a designated space.
					
How test will be performed: Select different available parking spot as a
destination for the navigation especially those edge cases spots near fire lanes
and drive ways. Make sure the suggested spot actually exists, and is not in a
space that only resembles a parking space.
					
\end{enumerate}

\paragraph{Robustness or Fault-Tolerance Requirements}

\begin{enumerate}

\item{NFR-PE4-T1}

Type: Structural, Dynamic, Manual
					
Initial State: The back-end service is disabled.
					
Input/Condition: System is launched.
					
Output/Result: The most recent parking spot information before back-end service
is disabled should be displayed.
					
How test will be performed: The developers will disable the API in the back-end
service that is responsible for sending spot information back to the system.
Then the system is launched which sends a request to acquire parking spot
information. Since the remote cache should store the information of the most
recent parking spot information, the system should still receive the most recent
information from the cache.
					
\end{enumerate}

\paragraph{Capacity Requirements}

\begin{enumerate}

\item{NFR-PE5-T1}

Type: Structural, Dynamic, Manual
					
Initial State: The system is operating normally.
					
Input/Condition: Sending $\hyperlink{concurrent_users}{CONCURRENT\_USERS}$
concurrent requests.
					
Output/Result: The system must be able to handle the
$\hyperlink{concurrent_users}{CONCURRENT\_USERS}$ different request with a valid
response
					
How test will be performed: The developers will create
$\hyperlink{concurrent_users}{CONCURRENT\_USERS}$ different API requests with
different parameters that requests different parking lot information. Then
relevant API testing frameworks would be used to send
\\$\hyperlink{concurrent_users}{CONCURRENT\_USERS}$ different web requests at
the same time. The developers should observe the system critical metrics such as
CPU utilization rate and RAM usage and relevant API response time on the
dashboard.
					
\end{enumerate}

\paragraph{Scalability or Extensibility Requirements}

\begin{enumerate}

\item{NFR-PE6-T1}

Type: Functional, Dynamic, Manual
					
Initial State: System has at least $\hyperlink{parking_lots}{PARKING\_LOTS}$
parking lot layout in \\$\hyperlink{timeframe}{TIMEFRAME}$ and is launched in a
vehicle.
					
Input/Condition: Valid addresses of $\hyperlink{parking_lots}{PARKING\_LOTS}$
parking lots and an empty parking space for each.
					
Output/Result: Correct directions and information on the parking lot is
displayed.
					
How test will be performed: The tester should enter a valid address of a parking
lot, select an empty parking space, and follow the directions provided by the
application. Upon successful completion, the tester should enter a different
valid address of a parking lot, select an empty parking space, and follow the
directions provided by the application. The success of the test is determined if
the tester is able to successfully navigate to each empty parking space located
in $\hyperlink{parking_lots}{PARKING\_LOTS}$ different parking lots.
					
\end{enumerate}

\subsubsection{Operational and Environmental Requirements}
\label{sec:5.2.4}
\paragraph{Expected Physical Environment}

\begin{enumerate}
\item{NFR-OE1-T1}

Type: Functional, Dynamic, Manual
					
Initial State: System is opened on a mobile device in a vehicle
					
Input/Condition: Input an address, select a parking space, and follow the
directions given by the application to the parking space.
			
Output/Result: The tester is able to complete all of the tasks in a vehicle and
is able to park successfully.
					
How test will be performed: A tester will be in a vehicle, opens the system on
their device and inputs a parking lot address and a space to navigate to. The
tester will then start the vehicle and navigate to the parking space with the
directions given by the application. The test will complete when the tester is
parked in the parking space.
\end{enumerate}

\paragraph{Release Requirements}

\begin{enumerate}
\item{NFR-OE2-T1}

Type: Structural, Dynamic, Manual
					
Initial State: N/A
					
Input/Condition: N/A
					
Output/Result: N/A
					
How test will be performed: Every month, the developers will check the GitHub
commit log for a new update.
\end{enumerate}

\subsubsection{Maintainability and Support Requirements}
\label{sec:5.2.5}
\paragraph{Maintenance Requirements}

\begin{enumerate}

\item{NFR-MA1-T1}

Type: Functional, Dynamic, Manual
					
Initial State: System is running in its normal state
					
Input/Condition: Either the software or hardware component of the system will go
under maintenance.
					
Output/Result: Component that is not under maintenance does not throw an
unexpected error.
					
How test will be performed: The developers will take the software component down
for maintenance and check that the hardware component does not throw an
unexpected error or fail, and vice versa with taking down the hardware component
down and checking the software component for unexpected errors.
					
\end{enumerate}

\paragraph{Supportability Requirements}

\begin{enumerate}

\item{NFR-MA2-T1}

Type: Functional, Dynamic, Manual
					
Initial State: System is running on the user's device
					
Input/Condition: N/A
					
Output/Result: N/A
					
How test will be performed: Random users will be surveyed. Users will be give 5
minutes to look through instructions provided in the interface and answer the
survey if they are satisfied with the instructions and contacts provided. At
least\\ $\hyperlink{supportability_satisfaction}{SUPPORTABILITY\_SATISFACTION}$
of surveyed users should be satisfied with the instructions and contacts
provided. \color{red}{At least 5 users will be surveyed. At least 80\% of users
surveyed should be satisfied with the instructions given.}
					
\end{enumerate}

%%%%%%%%%%%%%%%%%%% SECURITY REQUIREMENTS %%%%%%%%%%%%%%%%%%%
\subsubsection{Security Requirements}
\label{sec:5.2.6}
%%%%%%%%%%%%%%%%%%% ACCESS REQUIREMENTS %%%%%%%%%%%%%%%%%%%
\paragraph{Access Requirements}

\begin{enumerate}
%%%%%%%%%%%%%%%%%%% SR-1 %%%%%%%%%%%%%%%%%%%
\item{NFR-SR1-T1}

Type: Functional, Dynamic, Automatic
					
Initial State: Administrative website is launched and a login screen is
presented.
					
Input/Condition: Valid username and password for the parking lot owners/team.
					
Output/Result: Redirection to the administrative console with proper parking lot
data of their parking lot.
					
How test will be performed: A Cypress automation script will be created that
will input a valid username and password and check to ensure that the login is
successful and that the proper parking lot data of their parking lot is
displayed.

\item{NFR-SR1-T2}

Type: Functional, Dynamic, Automatic
					
Initial State: Administrative website is launched and a login screen is
presented.
					
Input/Condition: Invalid username and password for parking lot owners/team.
					
Output/Result: Prompt indicating that the login was unsuccessful.
					
How test will be performed: A Cypress automation script will be created that
will input an invalid username and password and check to ensure that the login
was unsuccessful, no redirection occurs, and that a prompt is shown that the
login was unsuccessful.

\item{NFR-SR1-T3}

Type: Functional, Dynamic, Automatic
					
Initial State: Driver application is launched.
					
Input/Condition: Address of a parking lot.
					
Output/Result: Unable to see analytics and parking lot data.
					
How test will be performed: A Cypress automation script will be created that
will input a valid address of a parking lot and ensure that analytics and
parking lot data are not displayed.

%%%%%%%%%%%%%%%%%%% SR-2 %%%%%%%%%%%%%%%%%%%
\item{NFR-SR2-T1}

Type: Functional, Dynamic, Automatic
					
Initial State: Logged in to the administrative console 
					
Input/Condition: Edit function is enabled and is selected for the designated
parking lot.
					
Output/Result: Edit view is displayed allowing for editing of the designated
parking lot.
					
How test will be performed: A Cypress automation script will be created that
will select the edit option and ensure that all the editing options are
available for the designated parking lot.

\item{NFR-SR2-T2}

Type: Functional, Dynamic, Automatic
					
Initial State: Driver application is launched.
					
Input/Condition: Address of a parking lot.
					
Output/Result: Edit function is not shown.
					
How test will be performed: A Cypress automation script will be created that
will input a valid address of a parking lot and ensure that no editing
functionality is enabled.

%%%%%%%%%%%%%%%%%%% SR-3 %%%%%%%%%%%%%%%%%%%
\item{NFR-SR3-T1}

Type: Functional, Dynamic, Automatic
					
Initial State: Logged in to the administrative console 
					
Input/Condition: Search for a parking lot that is not the designated parking
lot.
					
Output/Result: Edit functionality disabled and analytics are not displayed.

How test will be performed: A Cypress automation script will be created that
will input a valid address of a parking lot that is not the designated parking
lot and ensure that no editing functionality is enabled and analytics are not
displayed for the parking lot.

\end{enumerate}

\paragraph{Integrity Requirements}

\begin{enumerate}

%%%%%%%%%%%%%%%%%%% SR-4 %%%%%%%%%%%%%%%%%%%
\item{NFR-SR4-T1}

Type: Structural, Dynamic, Manual
					
Initial State: Logged into the database on the server
					
Input/Condition: Change the status of an empty parking lot entry in the database
to full and refresh the database
					
Output/Result: The database entry would revert the status of the parking lot
entry to the correct one, saying it is empty.

How test will be performed: The developers would enter the database and change
an empty parking lot entry in the database to full, refresh the database, and
ensure that the entry was reverted back to the correct status, which is empty.

%%%%%%%%%%%%%%%%%%% SR-5 %%%%%%%%%%%%%%%%%%%
\item{NFR-SR5-T1}

Type: Functional, Dynamic, Manual
					
Initial State: Logged in to the Administrative console
					
Input/Condition: Edit and save the parking lot layout and disconnect the system
from the internet.
					
Output/Result: Parking lot layout is cached and a prompt is shown indicating
that the unsaved layout is cached and the location.

How test will be performed: The tester will manually login to the administrative
console with proper credentials, edit the designated parking lot, and disconnect
from the internet. The tester will then see a prompt with saying the layout was
successfully cached locally and the link to the location.

%%%%%%%%%%%%%%%%%%% SR-6 %%%%%%%%%%%%%%%%%%%
\item{NFR-SR6-T1}

Type: Functional, Dynamic, Manual
					
Initial State: Logged in to the Administrative console
					
Input/Condition: Edit and save the parking lot layout and disconnect the system
from the internet.
					
Output/Result: System attempts to upload a cached parking lot layout every
\\$\hyperlink{attempt_upload_time}{ATTEMPT\_UPLOAD\_TIME}$ seconds and provides
a message that includes the countdown till the next attempt to upload.

How test will be performed: The tester will manually login to the administrative
console with proper credentials, edit and save the designated parking lot, and
disconnect from the internet. The tester will then check the developer console
for HTTP requests every $\hyperlink{attempt_upload_time}{ATTEMPT\_UPLOAD\_TIME}$
seconds to upload to the server as well as a message that provides the countdown
till the next upload.

%%%%%%%%%%%%%%%%%%% SR-7 %%%%%%%%%%%%%%%%%%%
\item{NFR-SR7-T1}

Type: Structural, Dynamic, Automatic
					
Initial State: Logged into the database on the server
					
Input/Condition: Check the last time the parking lot layouts on the server have
been backed up.
					
Output/Result: Each entry must have been backed up at 11:59PM daily.

How test will be performed: An automated pipeline would be created and ran daily
to facilitate the checking of each of the parking lot layouts.

%%%%%%%%%%%%%%%%%%% SR-8 %%%%%%%%%%%%%%%%%%%
\item{NFR-SR8-T1}

Type: Structural, Dynamic, Automatic
					
Initial State: Logged into the database on the server
					
Input/Condition: Attempt to add an entry with a different format in the database
from other parking spaces.
					
Output/Result: The database would produce an error and deny the addition of the
inconsistent entry.

How test will be performed: An automated test script will be created to test the
data consistency of the database through attempting to add an invalid entry and
ensuring that an error is produced and the addition is denied.

%%%%%%%%%%%%%%%%%%% SR-9 %%%%%%%%%%%%%%%%%%%
\item{NFR-SR9-T1}

Type: Structural, Dynamic, Automatic
					
Initial State: Logged into the database on the server
					
Input/Condition: Attempt to add an entry with more than \\
$\hyperlink{max_special_property}{MAXIMUM\_SPECIAL\_PROPERTY}$ special property.
					
Output/Result: The database would produce an error and deny the addition of the
inconsistent entry.

How test will be performed: An automated test script will be created to test the
data consistency of the database through attempting to add an entry with more
than $\hyperlink{max_special_property}{MAXIMUM\_SPECIAL\_PROPERTY}$ special
property and ensuring that an error is produced and the addition is denied.

%%%%%%%%%%%%%%%%%%% SR-10 %%%%%%%%%%%%%%%%%%%
\item{NFR-SR10-T1}

Type: Structural, Dynamic, Manual
					
Initial State: Database is offline and logged in to the Administrative console
					
Input/Condition: Attempt to edit and save the parking lot layout.
					
Output/Result: Parking lot layout is cached and a prompt is shown indicating
that the changes are unable to be uploaded to the database and that it is cached
locally.

How test will be performed: The developers would have to manually change the
database to offline, log into the administrative console, edit a parking lot
layout, and observe to see if a prompt is shown indicating that the changes are
unable to be uploaded to the database and that it is cached locally.

%%%%%%%%%%%%%%%%%%% SR-11 %%%%%%%%%%%%%%%%%%%
\item{NFR-SR11-T1}

Type: Functional, Dynamic, Automatic
					
Initial State: Logged in to the Administrative console
					
Input/Condition: Edit a parking lot layout and prompt the upload of the layout
to the database server.
					
Output/Result: Initiates the upload of the parking lot layout, prompts the user
that an upload is underway, and a prompt will appear upon success indicating
that the layout is successfully uploaded.

How test will be performed: A Cypress automation script will be created that
will edit a parking lot layout and prompt the upload of the layout to the
database server. It will verify that the upload was initiated and completed
successfully along with the prompts that the upload is underway and that the
layout is successfully uploaded respectively.

%%%%%%%%%%%%%%%%%%% SR-12 %%%%%%%%%%%%%%%%%%%
\item{NFR-SR12-T1}

Type: Functional, Dynamic
					
Initial State: Logged into the database on the server, download the parking lot
layout data represented in terms of a weighted undirected graph
					
Input/Condition: Find the shortest path between a given starting node to a
designated end node
					
Output/Result: The return value should be the set of edges from the starting
node to the end node
					
How test will be performed: The parking lot layout data is retrieved from the
database and reconstructed as a weighted undirected graph. Then the developers
dynamically test the connection and path between a starting node and designated
node by running a path finding algorithm such as Breath-first search.
%%%%%%%%%%%%%%%%%%% SR-13 %%%%%%%%%%%%%%%%%%%
\item{NFR-SR13-T1}

Type: Functional, Dynamic, Manual
					
Initial State: System is operating, but the navigation request is set to be
rejected by the back-end service
					
Input/Condition: User requests navigation
					
Output/Result: The system prompts a message indicating an error has occured.
					
How test will be performed: The navigation request is set to be rejected by the
back-end services by the developer. The developer then can test whether a error
message prompt appears when it cannot get the response from the server.

\end{enumerate}
\paragraph{Privacy Requirements}

\begin{enumerate}

\item{NFR-SR14-T1}

Type: Functional, Dynamic, Manual
					
Initial State: System is running on the user's device
					
Input/Condition: N/A
					
Output/Result: N/A
					
How test will be performed: Testers will check that a form pops up on the first
time launching the system to ask for permission to use the user's location.
					
\end{enumerate}

\subsubsection{Legal Requirements}
\label{sec:5.2.7}
\paragraph{Compliance Requirements}

\begin{enumerate}

\item{NFR-LR1-T1}

Type: Functional, Dynamic, Manual
					
Initial State: The system is operating
					
Input/Condition: The user is in navigation mode and while moving, the user
attempt to interact with the website
					
Output/Result: A notification window should pop out, informing the driver to
follow the regulations
					
How test will be performed: Developer will sit in the passenger seats with
navigation on, while the other developer will drive the vehicle. The developer
on the passenger seat will attempt to use the website but is prompted to follow
the regulations.
					
\end{enumerate}

\subsubsection{Health and Safety Requirements}
\label{sec:5.2.8}
\paragraph{Compliance Requirements}

\begin{enumerate}

\item{NFR-HS1-T1}

Type: Functional, Dynamic, Manual
					
Initial State: System is operating.
					
Input/Condition: User navigates between web pages from the system
					
Output/Result: The browser renders different web pages from the system
					
How test will be performed: Developer will manually test and navigate between
different web pages from the system and make sure no flashing graphics or
disturbing imagery exists
					
\end{enumerate}

\subsection{Traceability Between Test Cases and Requirements}
\wss{Provide a table that shows which test cases are supporting which
  requirements.} Included in this section are traceability matrices that maps
  test cases to functional requirements and non-functional requirements outlined
  in the
  \href{https://github.com/parkd-app/park-d/blob/main/docs/SRS/SRS.pdf}{SRS}.
\newpage
\begin{landscape}
\begin{table}[htbp]
\caption{Traceability Matrix for Test Cases and Functional Requirements - Part
1} \label{traceMatrix1}
\begin{tabularx}{\textwidth}{cc|c|c|c|c|c|c|c|c|c|c|c|c|c|}
\cline{3-15}
& & \multicolumn{13}{ c|}{Functional Requirements} \\ \cline{3-15} & & FR1  &
FR2 & FR3 & FR4 & FR5 & FR6 & FR7 & FR8 & FR9 & FR10 & FR11 & FR12 & FR13  \\
\cline{1-15} \multicolumn{1}{ |c| }{\multirow{13}{*}{Test Cases} } &
\multicolumn{1}{|c| } {BE1-FR1-T1}   &X&&&&&&&&&&&& \\ \cline{2-15}
\multicolumn{1}{|c| }{} 	                  & \multicolumn{1}{|c|
}{BE1-FR2-T1} &&X&&&&&&&&&&& \\ \cline{2-15} \multicolumn{1}{|c| }{} &
\multicolumn{1}{|c| } {BE1-FR3-T1}   &&&X&&&&&&&&&&\\ \cline{2-15}
\multicolumn{1}{|c| }{}                        & \multicolumn{1}{ |c| }
{BE2-FR4-T1}  &&&&X&&&&&&&&& \\ \cline{2-15} \multicolumn{1}{|c| }{} &
\multicolumn{1}{ |c| } {BE2-FR5-T1}  &&&&&X&&&&&&&&\\ \cline{2-15}
\multicolumn{1}{|c| }{}                        & \multicolumn{1}{ |c| }
{BE2-FR6-T1}  &&&&&&X&&&&&&& \\ \cline{2-15} \multicolumn{1}{|c| }{} &
\multicolumn{1}{ |c| } {BE2-FR7-T1}  &&&&&&&X&&&&&& \\ \cline{2-15}
\multicolumn{1}{|c| }{}                        & \multicolumn{1}{ |c| }
{BE2-FR8-T1}  &&&&&&&&X&&&&& \\ \cline{2-15} \multicolumn{1}{|c| }{} &
\multicolumn{1}{ |c| } {BE3-FR9-T1}  &&&&&&&&&X&&&&\\ \cline{2-15}
\multicolumn{1}{|c| }{}                        & \multicolumn{1}{ |c| }
{BE3-FR10-T1} &&&&&&&&&&X&&& \\ \cline{2-15} \multicolumn{1}{|c| }{} &
\multicolumn{1}{ |c| } {BE4-FR11-T1} &&&&&&&&&&&X&& \\ \cline{2-15}
\multicolumn{1}{|c| }{}                        & \multicolumn{1}{ |c| }
{BE4-FR12-T1} &&&&&&&&&&&&X& \\ \cline{2-15} \multicolumn{1}{|c| }{} &
\multicolumn{1}{ |c| } {BE4-FR13-T1} &&&&&&&&&&&&&X \\ \cline{1-15}
\end{tabularx}
\end{table}

\newpage
\begin{table}[htbp]
\caption{Traceability Matrix for Test Cases and Functional Requirements - Part
2} \label{traceMatrix1}
\begin{tabularx}{\textwidth}{cc|c|c|c|c|c|c|c|c|c|c|c|c|}
\cline{3-14}
& & \multicolumn{12}{ c|}{Functional Requirements} \\ \cline{3-14} & & FR14  &
FR15 & FR16 & FR17 & FR18 & FR19 & FR20 & FR21 & FR22 & FR23 & FR24 & FR25  \\
\cline{1-14} \multicolumn{1}{ |c| }{\multirow{12}{*}{Test Cases} } &
\multicolumn{1}{|c| } {BE4-FR14-T1}   &X&&&&&&&&&&& \\ \cline{2-14}
\multicolumn{1}{|c| }{} 	                  & \multicolumn{1}{|c|
}{BE5-FR15-T1} &&X&&&&&&&&&& \\ \cline{2-14} \multicolumn{1}{|c| }{} &
\multicolumn{1}{|c| } {BE5-FR16-T1}   &&&X&&&&&&&&&\\ \cline{2-14}
\multicolumn{1}{|c| }{}                        & \multicolumn{1}{ |c| }
{BE6-FR17-T1}  &&&&X&&&&&&&& \\ \cline{2-14} \multicolumn{1}{|c| }{} &
\multicolumn{1}{ |c| } {BE6-FR18-T1}  &&&&&X&&&&&&&\\ \cline{2-14}
\multicolumn{1}{|c| }{}                        & \multicolumn{1}{ |c| }
{BE6-FR19-T1}  &&&&&&X&&&&&& \\ \cline{2-14} \multicolumn{1}{|c| }{} &
\multicolumn{1}{ |c| } {BE6-FR20-T1}  &&&&&&&X&&&&& \\ \cline{2-14}
\multicolumn{1}{|c| }{}                        & \multicolumn{1}{ |c| }
{BE6-FR21-T1}  &&&&&&&&X&&&& \\ \cline{2-14} \multicolumn{1}{|c| }{} &
\multicolumn{1}{ |c| } {BE6-FR22-T1}  &&&&&&&&&X&&&\\ \cline{2-14}
\multicolumn{1}{|c| }{}                        & \multicolumn{1}{ |c| }
{BE6-FR23-T1} &&&&&&&&&&X&& \\ \cline{2-14} \multicolumn{1}{|c| }{} &
\multicolumn{1}{ |c| } {BE7-FR24-T1} &&&&&&&&&&&X& \\ \cline{2-14}
\multicolumn{1}{|c| }{}                        & \multicolumn{1}{ |c| }
{BE7-FR25-T1} &&&&&&&&&&&&X \\ \cline{1-14}
\end{tabularx}
\end{table}

\newpage
% \begin{table}[htbp] \caption{Traceability Matrix for Test Cases and Functional
% Requirements - Part 3} \label{traceMatrix1}
% \begin{tabularx}{\textwidth}{cc|c|c|c|c|c|c|c|c|c|c|c|c|} \cline{3-14} & &
% \multicolumn{6}{ c|}{Functional Requirements} \\ \cline{3-8} & & FR26  & FR27
% & FR28 & FR29 & FR30 & FR31 & FR32 & FR33 & FR34 & FR35 & FR36 & FR37 \\
% \cline{1-8} \multicolumn{1}{ |c| }{\multirow{6}{*}{Test Cases} } &
% \multicolumn{1}{|c| } {BE7-FR26-T1}   &X&&&&& \\ \cline{2-8}
% \multicolumn{1}{|c| }{}                      & \multicolumn{1}{|c|
% }{BE10-FR27-T1}    &&X&&&& \\ \cline{2-8} \multicolumn{1}{|c| }{} &
% \multicolumn{1}{|c| } {BE10-FR28-T1}   &&&X&&&\\
% \cline{2-8} \multicolumn{1}{|c| }{}                        & \multicolumn{1}{
% |c| } {BE10-FR29-T1}  &&&&X&& \\
% \cline{2-8} \multicolumn{1}{|c| }{} & \multicolumn{1}{ |c| } {BE11-FR30-T1}
% &&&&&X&\\ \cline{2-8} \multicolumn{1}{|c| }{}                        &
% \multicolumn{1}{ |c| } {BE11-FR31-T1}  &&&&&&X \\ \cline{1-8} \end{tabularx}
% \end{table}

\begin{table}[htbp]
\caption{Traceability Matrix for Test Cases and Functional Requirements - Part
3} \label{traceMatrix1}
\begin{tabularx}{\textwidth}{cc|c|c|c|c|c|c|c|c|c|c|c|c|}
\cline{3-14}
& & \multicolumn{12}{ c|}{Functional Requirements} \\ \cline{3-14} & & FR26  &
FR27 & FR28 & FR29 & FR30 & FR31 & FR32 & FR33 & FR34 & FR35 & FR36 & FR37  \\
\cline{1-14} \multicolumn{1}{ |c| }{\multirow{12}{*}{Test Cases} } &
\multicolumn{1}{|c| } {BE7-FR26-T1}   &X&&&&&&&&&&& \\ \cline{2-14}
\multicolumn{1}{|c| }{} 	                  & \multicolumn{1}{|c|
}{BE8-FR27-T1} &&X&&&&&&&&&& \\ \cline{2-14} \multicolumn{1}{|c| }{} &
\multicolumn{1}{|c| } {BE8-FR28-T1}   &&&X&&&&&&&&&\\ \cline{2-14}
\multicolumn{1}{|c| }{}                        & \multicolumn{1}{ |c| }
{BE9-FR29-T1}  &&&&X&&&&&&&& \\ \cline{2-14} \multicolumn{1}{|c| }{} &
\multicolumn{1}{ |c| } {BE9-FR30-T1}  &&&&&X&&&&&&&\\ \cline{2-14}
\multicolumn{1}{|c| }{}                        & \multicolumn{1}{ |c| }
{BE10-FR31-T1}  &&&&&&X&&&&&& \\ \cline{2-14} \multicolumn{1}{|c| }{} &
\multicolumn{1}{ |c| } {BE10-FR32-T1}  &&&&&&&X&&&&& \\ \cline{2-14}
\multicolumn{1}{|c| }{}                        & \multicolumn{1}{ |c| }
{BE10-FR33-T1}  &&&&&&&&X&&&& \\ \cline{2-14} \multicolumn{1}{|c| }{} &
\multicolumn{1}{ |c| } {BE11-FR34-T1}  &&&&&&&&&X&&&\\ \cline{2-14}
\multicolumn{1}{|c| }{}                        & \multicolumn{1}{ |c| }
{BE11-FR35-T1} &&&&&&&&&&X&& \\ \cline{2-14} \multicolumn{1}{|c| }{} &
\multicolumn{1}{ |c| } {BE12-FR36-T1} &&&&&&&&&&&X& \\ \cline{2-14}
\multicolumn{1}{|c| }{}                        & \multicolumn{1}{ |c| }
{BE12-FR37-T1} &&&&&&&&&&&&X \\ \cline{1-14}
\end{tabularx}
\end{table}

\newpage
\begin{table}[htbp]
\caption{Traceability Matrix for Test Cases and Non-Functional Requirements -
Look \& Feel, Usability \& Humanity, and Performance} \label{traceMatrix1}
\begin{tabularx}{\textwidth}{cc|c|c|c|c|c|c|c|c|c|c|c|c|}
\cline{3-14}
& & \multicolumn{12}{ c|}{Non-Functional Requirements} \\ \cline{3-14} & & LF1 &
LF2 & UH1 & UH2 & UH3 & UH4 & PE1 & PE2 & PE3 & PE4 & PE5 & PE6 \\
\cline{1-14} \multicolumn{1}{ |c| }{\multirow{13}{*}{Test Cases} } &
\multicolumn{1}{|c| } {NFR-LF1-T1}   &X&&&&&&&&&&& \\ \cline{2-14}
\multicolumn{1}{|c| }{} 	                  & \multicolumn{1}{|c|
}{NFR-LF2-T1} &&X&&&&&&&&&& \\ \cline{2-14} \multicolumn{1}{|c| }{} &
\multicolumn{1}{|c| } {NFR-UH1-T1}   &&&X&&&&&&&&&\\ \cline{2-14}
\multicolumn{1}{|c| }{}                        & \multicolumn{1}{ |c| }
{NFR-UH2-T1}  &&&&X&&&&&&&& \\ \cline{2-14} \multicolumn{1}{|c| }{} &
\multicolumn{1}{ |c| } {NFR-UH3-T1}  &&&&&X&&&&&&&\\ \cline{2-14}
\multicolumn{1}{|c| }{}                        & \multicolumn{1}{ |c| }
{NFR-UH4-T1}  &&&&&&X&&&&&& \\ \cline{2-14} \multicolumn{1}{|c| }{} &
\multicolumn{1}{ |c| } {NFR-PE1-T1}  &&&&&&&X&&&&& \\ \cline{2-14}
\multicolumn{1}{|c| }{}                        & \multicolumn{1}{ |c| }
{NFR-PE2-T1}  &&&&&&&&X&&&& \\ \cline{2-14} \multicolumn{1}{|c| }{} &
\multicolumn{1}{ |c| } {NFR-PE3-T1}  &&&&&&&&&X&&&\\ \cline{2-14}
\multicolumn{1}{|c| }{}                        & \multicolumn{1}{ |c| }
{NFR-PE4-T1} &&&&&&&&&&X&& \\ \cline{2-14} \multicolumn{1}{|c| }{} &
\multicolumn{1}{ |c| } {NFR-PE5-T1} &&&&&&&&&&&X& \\ \cline{2-14}
\multicolumn{1}{|c| }{}                        & \multicolumn{1}{ |c| }
{NFR-PE6-T1} &&&&&&&&&&&&X \\ \cline{1-14}
\end{tabularx}
\end{table}

\newpage
\begin{table}[htbp]
\caption{Traceability Matrix for Test Cases and Non-Functional Requirements -
Operational \& Environmental and Maintainability \& Support}
\label{traceMatrix1}
\begin{tabularx}{\textwidth}{cc|c|c|c|c|}
\cline{3-6}
& & \multicolumn{4}{ c|}{Non-Functional Requirements} \\ \cline{3-6} & & OE1  &
OE2 & MA1 & MA2  \\ \cline{1-6} \multicolumn{1}{ |c| }{\multirow{4}{*}{Test
Cases} } & \multicolumn{1}{|c| } {NFR-OE1-T1}   &X&&& \\ \cline{2-6}
\multicolumn{1}{|c| }{} 	                  & \multicolumn{1}{|c|
}{NFR-OE2-T1} &&X&& \\ \cline{2-6} \multicolumn{1}{|c| }{} & \multicolumn{1}{|c|
} {NFR-MA1-T1}   &&&X&\\ \cline{2-6} \multicolumn{1}{|c| }{} & \multicolumn{1}{
|c| } {NFR-MA2-T1}  &&&&X\\ \cline{1-6}
\end{tabularx}
\end{table}

\newpage
\begin{table}[htbp]
\caption{Traceability Matrix for Test Cases and Non-Functional Requirements -
Security} \label{traceMatrix1}
\begin{tabularx}{\textwidth}{cc|c|c|c|c|c|c|c|c|c|c|c|c|c|c|}
\cline{3-16}
& & \multicolumn{14}{ c|}{Non-Functional Requirements} \\ \cline{3-16} & & SR1 &
SR2 & SR3 & SR4 & SR5 & SR6 & SR7 & SR8 & SR9 & SR10 & SR11 & SR12 & SR13 & SR14
\\ \cline{1-16} \multicolumn{1}{ |c| }{\multirow{14}{*}{Test Cases} } &
\multicolumn{1}{|c| } {NFR-SR1-T1}   &X&&&&&&&&&&&&& \\ \cline{2-16}
\multicolumn{1}{|c| }{} 	                  & \multicolumn{1}{|c|
}{NFR-SR1-T2} &X&&&&&&&&&&&&& \\ \cline{2-16} \multicolumn{1}{|c| }{} &
\multicolumn{1}{|c| } {NFR-SR1-T3}   &X&&&&&&&&&&&&&\\ \cline{2-16}
\multicolumn{1}{|c| }{}                        & \multicolumn{1}{ |c| }
{NFR-SR2-T1}  &&X&&&&&&&&&&&& \\ \cline{2-16} \multicolumn{1}{|c| }{} &
\multicolumn{1}{ |c| } {NFR-SR2-T2}  &&X&&&&&&&&&&&&\\ \cline{2-16}
\multicolumn{1}{|c| }{}                        & \multicolumn{1}{ |c| }
{NFR-SR3-T1}  &&&X&&&&&&&&&&& \\ \cline{2-16} \multicolumn{1}{|c| }{} &
\multicolumn{1}{ |c| } {NFR-SR4-T1}  &&&&X&&&&&&&&&& \\ \cline{2-16}
\multicolumn{1}{|c| }{}                        & \multicolumn{1}{ |c| }
{NFR-SR5-T1}  &&&&&X&&&&&&&&& \\ \cline{2-16} \multicolumn{1}{|c| }{} &
\multicolumn{1}{ |c| } {NFR-SR6-T1}  &&&&&&X&&&&&&&&\\ \cline{2-16}
\multicolumn{1}{|c| }{}                        & \multicolumn{1}{ |c| }
{NFR-SR7-T1} &&&&&&&X&&&&&&& \\ \cline{2-16} \multicolumn{1}{|c| }{} &
\multicolumn{1}{ |c| } {NFR-SR8-T1} &&&&&&&&X&&&&&& \\ \cline{2-16}
\multicolumn{1}{|c| }{}                        & \multicolumn{1}{ |c| }
{NFR-SR9-T1} &&&&&&&&&X&&&&& \\ \cline{2-16} \multicolumn{1}{|c| }{} &
\multicolumn{1}{ |c| } {NFR-SR10-T1} &&&&&&&&&&X&&&& \\ \cline{2-16}
\multicolumn{1}{|c| }{}                        & \multicolumn{1}{ |c| }
{NFR-SR11-T1} &&&&&&&&&&&X&&& \\ \cline{2-16} \multicolumn{1}{|c| }{} &
\multicolumn{1}{ |c| } {NFR-SR12-T1} &&&&&&&&&&&&X&& \\ \cline{2-16}
\multicolumn{1}{|c| }{}                        & \multicolumn{1}{ |c| }
{NFR-SR13-T1} &&&&&&&&&&&&&X& \\ \cline{2-16} \multicolumn{1}{|c| }{} &
\multicolumn{1}{ |c| } {NFR-SR14-T1} &&&&&&&&&&&&&&X \\ \cline{1-16}
\end{tabularx}
\end{table}

\newpage
\begin{table}[htbp]
\caption{Traceability Matrix for Test Cases and Non-Functional Requirements -
Legal and Health \& Safety} \label{traceMatrix1}
\begin{tabularx}{\textwidth}{cc|c|c|c|c|}
\cline{3-4}
& & \multicolumn{2}{ c|}{Non-Functional Requirements} \\ \cline{3-4} & & LR1  &
HS1  \\ \cline{1-4} \multicolumn{1}{ |c| }{\multirow{2}{*}{Test Cases} } &
\multicolumn{1}{|c| } {NFR-LR1-T1}   &X& \\ \cline{2-4} \multicolumn{1}{|c| }{}
& \multicolumn{1}{|c| }{NFR-HS1-T1}    &&X \\ \cline{1-4}
\end{tabularx}
\end{table}
\end{landscape}


\section{Unit Test Description}

% \wss{Reference your MIS (detailed design document) and explain your overall
%   philosophy for test case selection.}  
% \wss{This section should not be filled in until after the MIS (detailed design
%   document) has been completed.}
For non-functional requirements, we will utilize a unit test library to perform
unit testing for this toolbox. To create unit tests for the backend portion of
the program, we will generate a corresponding test file for two key modules,
which will include unit tests for every function within the module. These tests
will consist of a range of inputs, including those that are essential for the
success of updating the parking spot availability in real time.
\textcolor{red}{For the frontend user interface portion, we created a couple of
test files with Cypress, a UI testing framework, one for each main page of the
application (i.e. Login Screen, User Mode, Admin View). These tests consists of
input validation logic, ensuring that the CSS of the UI elements are correctly
defined, and that certain elements are displayed on the screen dependent on
their user mode (if they are an administrator or a user).}\\

To evaluate the thoroughness of our testing, we will employ coverage metrics.
Specifically, we will utilize the coverage.py Python library for the backend,
which enables swift analysis of code coverage for all project modules. Our
target is to attain 80\% code coverage per module, which will guarantee that all
functions are tested comprehensively. For the frontend, we be using Istanbul,
which plays very well with the UI automation framework we are using and allows
us to get a full analysis of the code coverage. We wish to similarly achieve at
least 80\% line coverage for the frontend.
%\wss{Reference your MIS (detailed design document) and explain your overall
 % philosophy for test case selection.}  
%\wss{This section should not be filled in until after the MIS (detailed design
  %document) has been completed.}

\subsection{Unit Testing Scope}

Database module as well as parking stats module will be verified for correct
functionality (correct sample inputs output correct sample outputs).
\textcolor{red}{For the Authentication Module, it would be verified for
successful logins and error handling.} The machine learning module will not be
verified through unit testing as the model is obtained from DarkNet with
pre-trained layers.

\textcolor{red}{In terms of the importance of the modules for verification,
below is a table that ranks each module in the
\href{https://github.com/parkd-app/park-d/blob/main/docs/Design/SoftArchitecture/MG.pdf}{Module
Guide} and
\href{https://github.com/parkd-app/park-d/blob/main/docs/Design/SoftDetailedDes/MIS.pdf}{Module
Interface Specification} based on the priority for verification. We rank using
High, Medium, and Low.}

\begin{table}[H] 
    \centering
    \begin{tabularx}{\textwidth}{|X|X|}
        \toprule
        \textbf{Module} & \textbf{Priority}\\
        \midrule
        Video capture module & High\\
        View module & High\\
        Machine learning model module & High\\
        Database module & High\\
        Parking Stats module & Medium\\
        Admin console module & Medium\\
        Admin module &  Medium\\
        Parking lot layout module & Medium\\
        Parking layout element module & Medium\\
        Parking spot module & Medium\\
        Authentication module & Medium\\
        Navigation module & Medium\\
        User action handler module & Medium\\
        User module & Low\\
        Vehicle module & Low\\
        \bottomrule
    \end{tabularx}
\caption{Module Verification Priority} \label{tab:moduleVerificationPriority}
\end{table}


%\wss{What modules are outside of the scope.  If there are modules that are
%  developed by someone else, then you would say here if you aren't planning on
%  verifying them.  There may also be modules that are part of your software,
%  but have a lower priority for verification than others.  If this is the case,
%  5 explain your rationale for the ranking of module importance.}

\subsection{Tests for Functional Requirements}
These tests refer to functional requirements outlined in the
\href{https://github.com/parkd-app/park-d/blob/main/docs/SRS/SRS.pdf}{Software
Requirements Specification}. \subsubsection{\textcolor{red}{Database Module Test
Cases}} \textcolor{red}{
\vspace{0.5cm}
\hspace{0.5cm}
\begin{tabular}{|l|p{10cm}|}
    \hline
    \bf{Test} 6.2.1.1: & Saving admin annotation coordinates from the user
    interface into the database. \\
     \bf{Test file: } &
     \href{https://github.com/parkd-app/park-d/blob/main/test/TestDB.py}{src/test/testDB.py}\\
    \bf{Requirements}: &  FR5, FR6, FR7, FR36, FR37\\
    \bf{Description}: & Tests if the annotation coordinates for ML model is
    properly saved in the database with a identifier that is unique to a parking
    lot\\
    \bf{Type}: & Unit test (static, automated) \\
    \bf{Initial State}: & The user has enter the annotation page. \\
    \bf{Input}: & Json object includes the annotation coordinate under the
    "CamCoords" key. \\
    \bf{Output}: & Response body object \\
    \bf{Pass}: & The response body object contains the previously saved
    CamCoords data. \\
    \hline
\end{tabular}}


\vspace*{0.5 cm}


\textcolor{red}{
\begin{center}
\begin{tabular}{|l|p{10cm}|}
    \hline
    \bf{Test} 6.2.1.2: & Saving admin annotation coordinates from the user
    interface into the database. \\
     \bf{Test file: } &
     \href{https://github.com/parkd-app/park-d/blob/main/test/TestDB.py}{src/test/testDB.py}\\
    \bf{Requirements}: & FR5, FR6, FR7, FR8, FR36, FR37  \\
    \bf{Description}: & Tests if the annotation coordinates for user view is
    properly saved in the database with a identifier that is unique to a parking
    lot\\
    \bf{Type}: & Unit test (static, automated) \\
    \bf{Initial State}: & The user has enter the annotation page. \\
    \bf{Input}: & Json object includes the annotation coordinate under the
    "MapCoords" key. \\
    \bf{Output}: & Response body object \\
    \bf{Pass}: & The response body object contains the previously saved
    MapCoords data. \\
    \hline
\end{tabular}
\end{center}}


\vspace*{0.5 cm}


\textcolor{red}{
\begin{center}
\begin{tabular}{|l|p{10cm}|}
    \hline
    \bf{Test} 6.2.1.3: & Saving YouTube stream link \\
     \bf{Test file: } &
     \href{https://github.com/parkd-app/park-d/blob/main/test/TestDB.py}{src/test/testDB.py}\\
    \bf{Requirements}: & FR5, FR6, FR7, FR8 \\
    \bf{Description}: & Tests if the YouTube stream link with a parking lot is
    saved in the database with a identifier that is unique to a parking lot\\
    \bf{Type}: & Unit test (static, automated) \\
    \bf{Initial State}: &  \\
    \bf{Input}: & The user has enter the YouTube link. \\
    \bf{Output}: & Response body object \\
    \bf{Pass}: & The response body object contains the previously saved YouTube
    link data. \\
    \hline
\end{tabular}
\end{center}}


\vspace*{0.5 cm}


\textcolor{red}{
\begin{center}
\begin{tabular}{|l|p{10cm}|}
    \hline
    \bf{Test} 6.2.1.4: & Updating model input source \\
     \bf{Test file: } &
     \href{https://github.com/parkd-app/park-d/blob/main/test/TestML.py}{src/test/testML.py}\\
    \bf{Requirements}: & FR5, FR6, FR7, FR8, FR9, FR11 \\
    \bf{Description}: & Tests if the model input video can be correctly set\\
    \bf{Type}: & Unit test (static, automated) \\
    \bf{Initial State}: & Model is running as usual \\
    \bf{Input}: & The user enter the parking lot owner name and id \\
    \bf{Output}: & The current\_input key on the database \\
    \bf{Pass}: & The string URL under the current\_input key should be the same
    as the YouTube URL for the corresponding parking lot  \\
    \hline
\end{tabular}
\end{center}}


\vspace*{0.5 cm}


\textcolor{red}{
\begin{center}
\begin{tabular}{|l|p{10cm}|}
    \hline
    \bf{Test} 6.2.1.5: & Updating model input annotation coordinates \\
    \bf{Test file: } &
    \href{https://github.com/parkd-app/park-d/blob/main/test/TestML.py}{src/test/testML.py}\\
    \bf{Requirements}: & FR5, FR6, FR7, FR8, FR9, FR11 \\
    \bf{Description}: & Tests if the model can correctly retrieve the annotation
    coordinates\\
    \bf{Type}: & Unit test (static, automated) \\
    \bf{Initial State}: & Model is running as usual \\
    \bf{Input}: & The user enter the parking lot owner name and id \\
    \bf{Output}: & The current\_coordinate key on the database \\
    \bf{Pass}: & The JSON object under the current\_coordinate key should be the
    same as the coordinates for the corresponding parking lot annotations \\
    \hline
\end{tabular}
\end{center}}


\vspace*{0.5 cm}


\subsubsection{\textcolor{red}{Parking Stats Module Test Cases}}
\textcolor{red}{
\begin{center}
\begin{tabular}{|l|p{10cm}|}
    \hline
    \bf{Test} 6.2.2.1: & Analytics data updating \\
    \bf{Test file: } &
    \href{https://github.com/parkd-app/park-d/blob/main/test/TestAnaly.py}{src/test/testAnaly.py}\\
    \bf{Requirements}: & FR34, FR35 \\
    \bf{Description}: & Tests if hourly analytics data is successfully appended
    to the database\\
    \bf{Type}: & Unit test (static, automated) \\
    \bf{Initial State}: & Model is running \\
    \bf{Input}: & User append the analytic 1 hour data to the database \\
    \bf{Output}: & The up to date analytic data with a testing parking lot \\
    \bf{Pass}: & The returned analytic data should include the recently appended
    data of the parking lot \\
    \hline
\end{tabular}
\end{center}}

\subsubsection{\textcolor{red}{View Module}} \textcolor{red}{
\begin{center}
\begin{tabular}{|l|p{10cm}|}
    \hline
    \bf{Test} 6.2.3.1: & Successful Login \\
    \bf{Test file: } &
    \href{https://github.com/parkd-app/park-d/blob/main/src/park-d-front-end/src/cypress/e2e/Login_Test_Suite.cy.js}{cypress/e2e/Login\_Test\_Suite.cy.js}\\
    \bf{Requirements}: & FR31, FR32\\
    \bf{Description}: & Tests the successful login of an administrator\\
    \bf{Type}: & Unit test (static, automated) \\
    \bf{Initial State}: & Login screen is presented. \\
    \bf{Input}: & Valid email and password for the parking lot owners/team \\
    \bf{Output}: & Administrative console \\
    \bf{Pass}: &  Redirected to their respective parking lot administrative
    console\\
    \hline
\end{tabular}
\end{center}}

\textcolor{red}{
\begin{center}
\begin{tabular}{|l|p{10cm}|}
    \hline
    \bf{Test} 6.2.3.2: & Unsuccessful Login - Invalid Email \\
    \bf{Test file: } &
    \href{https://github.com/parkd-app/park-d/blob/main/src/park-d-front-end/src/cypress/e2e/Login_Test_Suite.cy.js}{cypress/e2e/Login\_Test\_Suite.cy.js}\\
    \bf{Requirements}: & FR31, FR32\\
    \bf{Description}: & Tests the unsuccessful login attempt of an administrator
    with an invalid email\\
    \bf{Type}: & Unit test (static, automated) \\
    \bf{Initial State}: & Login screen is presented. \\
    \bf{Input}: & Invalid email for the parking lot owners/team \\
    \bf{Output}: & Error message \\
    \bf{Pass}: &  An error message is shown indicating an invalid email was
    given and prompts to try again.\\
    \hline
\end{tabular}
\end{center}}

\textcolor{red}{
\begin{center}
\begin{tabular}{|l|p{10cm}|}
    \hline
    \bf{Test} 6.2.3.3: & Unsuccessful Login - No real user \\
    \bf{Test file: } &
    \href{https://github.com/parkd-app/park-d/blob/main/src/park-d-front-end/src/cypress/e2e/Login_Test_Suite.cy.js}{cypress/e2e/Login\_Test\_Suite.cy.js}\\
    \bf{Requirements}: & FR31, FR32\\
    \bf{Description}: & Tests the unsuccessful login attempt of an administrator
    with an email that does not exist \\
    \bf{Type}: & Unit test (static, automated) \\
    \bf{Initial State}: & Login screen is presented. \\
    \bf{Input}: & Non-existent email and password for the parking lot
    owners/team \\
    \bf{Output}: & Error message \\
    \bf{Pass}: &  An error message is shown indicating the email does not exist
    within the database and prompts to try again.\\
    \hline
\end{tabular}
\end{center}}

\textcolor{red}{
\begin{center}
\begin{tabular}{|l|p{10cm}|}
    \hline
    \bf{Test} 6.2.3.4: & Unsuccessful Login - No input \\
    \bf{Test file: } &
    \href{https://github.com/parkd-app/park-d/blob/main/src/park-d-front-end/src/cypress/e2e/Login_Test_Suite.cy.js}{cypress/e2e/Login\_Test\_Suite.cy.js}\\
    \bf{Requirements}: & FR31, FR32\\
    \bf{Description}: & Tests the unsuccessful login attempt of an administrator
    with no input \\
    \bf{Type}: & Unit test (static, automated) \\
    \bf{Initial State}: & Login screen is presented. \\
    \bf{Input}: &  N/A \\
    \bf{Output}: & Error message \\
    \bf{Pass}: &  An error message is shown indicating no input is given and
    prompts to try again.\\
    \hline
\end{tabular}
\end{center}}

\textcolor{red}{
\begin{center}
\begin{tabular}{|l|p{10cm}|}
    \hline
    \bf{Test} 6.2.3.5: & Unsuccessful Login - Wrong Password \\
    \bf{Test file: } &
    \href{https://github.com/parkd-app/park-d/blob/main/src/park-d-front-end/src/cypress/e2e/Login_Test_Suite.cy.js}{cypress/e2e/Login\_Test\_Suite.cy.js}\\
    \bf{Requirements}: & FR31, FR32\\
    \bf{Description}: & Tests the unsuccessful login attempt of an administrator
    with a valid email but incorrect password \\
    \bf{Type}: & Unit test (static, automated) \\
    \bf{Initial State}: & Login screen is presented. \\
    \bf{Input}: &  Valid email and incorrect password for the parking lot
    owners/team \\
    \bf{Output}: & Error message \\
    \bf{Pass}: &  An error message is shown indicating an incorrect password is
    given and prompts to try again.\\
    \hline
\end{tabular}
\end{center}}

\textcolor{red}{
\begin{center}
\begin{tabular}{|l|p{10cm}|}
    \hline
    \bf{Test} 6.2.3.6: & Redirection to User Mode \\
    \bf{Test file: } &
    \href{https://github.com/parkd-app/park-d/blob/main/src/park-d-front-end/src/cypress/e2e/Login_Test_Suite.cy.js}{cypress/e2e/Login\_Test\_Suite.cy.js}\\
    \bf{Requirements}: & FR31, FR32\\
    \bf{Description}: & Tests the proper redirection to User Mode \\
    \bf{Type}: & Unit test (static, automated) \\
    \bf{Initial State}: & Login screen is presented. \\
    \bf{Input}: &  Click button to navigate to User Mode\\
    \bf{Output}: & User Screen \\
    \bf{Pass}: &  Redirected to the user screen\\
    \hline
\end{tabular}
\end{center}}

\textcolor{red}{
\begin{center}
\begin{tabular}{|l|p{10cm}|}
    \hline
    \bf{Test} 6.2.3.7: & Admin privileges \\
    \bf{Test file: } &
    \href{https://github.com/parkd-app/park-d/blob/main/src/park-d-front-end/src/cypress/e2e/Admin_Test_Suite.cy.js}{cypress/e2e/Admin\_Test\_Suite.cy.js}\\
    \bf{Requirements}: & FR32, FR33, FR34, FR36, FR37\\
    \bf{Description}: & Test if an admin has all the proper privileges\\
    \bf{Type}: & Unit test (static, automated) \\
    \bf{Initial State}: & Admin is logged in\\
    \bf{Input}: &  Open sidebar\\
    \bf{Output}: & Sidebar opens\\
    \bf{Pass}: & Checks if all the correct features are included in the sidebar
    \\
    \hline
\end{tabular}
\end{center}}

\textcolor{red}{
\begin{center}
\begin{tabular}{|l|p{10cm}|}
    \hline
    \bf{Test} 6.2.3.8: & Admin Logout \\
    \bf{Test file: } &
    \href{https://github.com/parkd-app/park-d/blob/main/src/park-d-front-end/src/cypress/e2e/Admin_Test_Suite.cy.js}{cypress/e2e/Admin\_Test\_Suite.cy.js}\\
    \bf{Requirements}: & FR31, FR32\\
    \bf{Description}: & Test if an admin can logout\\
    \bf{Type}: & Unit test (static, automated) \\
    \bf{Initial State}: & Admin is logged in\\
    \bf{Input}: &  Click on logout button\\
    \bf{Output}: & Login Page\\
    \bf{Pass}: & Redirected to login page \\
    \hline
\end{tabular}
\end{center}}

\subsection{Tests for Nonfunctional Requirements}
Unit testing for nonfunctional requirements is not applicable for this project.
% \wss{If there is a module that needs to be independently assessed for
%   performance, those test cases can go here.  In some projects, planning for
%   nonfunctional tests of units will not be that relevant.}

% \wss{These tests may involve collecting performance data from previously
%   mentioned functional tests.}

% \subsubsection{Module ?}
		
% \begin{enumerate}

% \item{test-id1\\}

% Type: \wss{Functional, Dynamic, Manual, Automatic, Static etc. Most will be
%   automatic}
					
% Initial State: 
					
% Input/Condition: 
					
% Output/Result: 
					
% How test will be performed: 
					
% \item{test-id2\\}

% Type: Functional, Dynamic, Manual, Static etc.
					
% Initial State: 
					
% Input: 
					
% Output: 
					
% How test will be performed: 

% \end{enumerate}

% \subsubsection{Module ?}

% ...

\subsection{Traceability Between Test Cases and Modules}
\begin{table}[H]
\centering
\begin{tabular}{p{0.2\textwidth} p{0.6\textwidth}}
\toprule
\textbf{Test Number} & \textbf{Modules}\\
\midrule
6.2.1.1 & Database Module\\
6.2.1.2 & Database Module\\
6.2.1.3 & Database Module\\
6.2.1.4 & Database Module\\
6.2.1.5 & Database Module\\
\midrule
6.2.2.1 & Parking Stats Module\\
\midrule
6.2.3.1 & View Module\\
6.2.3.2 & View Module\\
6.2.3.3 & View Module\\
6.2.3.4 & View Module\\
6.2.3.5 & View Module\\
6.2.3.6 & View Module\\
6.2.3.7 & View Module\\
6.2.3.8 & View Module\\
\bottomrule
\end{tabular}
\caption{Trace Between Unit Test Cases and Modules}
\label{TblRT}
\end{table}

% \wss{Provide evidence that all of the modules have been considered.}
				
% \bibliographystyle{plainnat}

% \bibliography{../../refs/References}
\newpage

\section{Appendix}

Additional information that supports this documentation is provided in the
following sections.

\subsection{Symbolic Parameters}
\noindent $\hypertarget{max_taps}{MAX\_TAPS}$ = 2 \\
$\hypertarget{min_framerate}{MIN\_FRAMERATE}$ = 15\\
$\hypertarget{concurrent_users}{CONCURRENT\_USERS}$ = 10\\
$\hypertarget{parking_lots}{PARKING\_LOTS}$ = 2\\
$\hypertarget{default_delay}{DEFAULT\_DELAY}$ = 10\\
$\hypertarget{attempt_upload_time}{ATTEMPT\_UPLOAD\_TIME}$ = 30 \\
$\hypertarget{max_special_property}{MAXIMUM\_SPECIAL\_PROPERTY}$ = 1 \\
$\hypertarget{supportability_satisfaction}{SUPPORTABILITY\_SATISFACTION}$ = 80\%
\\
$\hypertarget{exploration_time}{EXPLORATION\_TIME}$ = 5 minutes\% \\
$\hypertarget{timeframe}{TIMEFRAME}$ = 6 months\% \\
$\hypertarget{timeout_tolerance}{TIMEOUT\_TOLERANCE}$ = 60 \\
$\hypertarget{distance_tolerance}{DISTANCE\_TOLERANCE}$ = 5 \\
$\hypertarget{driving_estimate_tolerance}{DRIVING\_ESTIMATE\_TOLERANCE}$ = 60 \\
\subsection{Usability Survey Questions?}
\label{sec:7.2}

\begin{enumerate}
    \item How responsive was the system in providing directions to your desired
    parking space? [1: Lowest, 5 Highest]
    \item How responsive was the system in providing new directions when your
    desired parking space became unavailable? [1: Lowest, 5 Highest]
    \item Were you able to reach your desired parking space using the given
    directions? [Yes/No]
    \item Is the system visually/stylistically similar to Google Maps, Apple
    Maps, or Waze? [Yes/No]
    \item Is the system as easy to use as Google Maps, Apple Maps, or Waze?
    [Yes/No]
    \item Were you able to use the system to find an available parking spot?
    [Yes/No]
    \item Were you satisfied with the instructions and contacts provided in the
    interface? [Yes/No]
    \item How often would you use this system when looking for parking spaces?
    [1: Very Infrequently, 2: Somewhat Infrequently, 3: Occasionally, 4:
    Somewhat Frequently, 5: Frequently]
\end{enumerate}


\newpage{}
\subsection{Reflection}

\wss{The information in this section will be used to evaluate the team members
on the graduate attribute of Lifelong Learning.  Please answer the following
questions:}


\begin{enumerate}
  \item \wss{What knowledge and skills will the team collectively need to
  acquire to successfully complete the verification and validation of your
  project? Examples of possible knowledge and skills include dynamic testing
  knowledge, static testing knowledge, specific tool usage etc.  You should look
  to identify at least one item for each team member.}

  \item \wss{For each of the knowledge areas and skills identified in the
  previous question, what are at least two approaches to acquiring the knowledge
  or mastering the skill?  Of the identified approaches, which will each team
  member pursue, and why did they make this choice?}
\end{enumerate}

\subsubsection{Manual Testing}
Despite the advent of automated testing tools, which are indeed useful, not all
features of our system can be tested this way. For example, to validate one of
our requirements, we need to test that a new set of directions are provided to
the user if the current set of directions are infeasible to follow. As such, we
need to test this feature by asking one of the testers to stand in front of the
vehicle and check that the driver receives a new set of instructions. This
verification cannot simply be done using an automated approach. \\
\\
\noindent \textbf{Learning Approaches}
\begin{enumerate}
    \item Observation Skills: It is inadequate to simply fail a test on the
    principle that the output did not match the expected result. In order to
    hone your manual testing skills, you must pay close attention to all of your
    actions and the corresponding system response, such that the failure can be
    reproduced in the future.
    \item Analytical Skills: To improve your manual testing skills, you must
    think critically about the reason why the test scenario failed because the
    system may have been in a state that it should have not been in.
    Furthermore, by being more analytical, you can think of more ways to break
    the system and thus, reduce the number of bugs and glitches.
\end{enumerate}

\begin{itemize}
    \item Almen: To improve my manual testing skills, I believe working on my
    analytical skills should be the priority. With such a complex system, having
    analytical skills would help with breaking it up into smaller components to
    gain better understanding of the components of the system and how they
    interact with each other and collecting data needed to make decisions and
    suggestions.
    \item Kabishan: I believe that I can improve my manual testing skills by
    mastering my analytical skills. In doing so, not only will I test our system
    to a higher depth, but I will also improve my ability to test my software
    solutions at work. I will improve my analytical skills by consulting my
    notes from 3S03, the Software Testing course, and attempting manual testing
    interview questions online.
    \item Jonathan: To further improve my manual testing skills, I believe that
    focusing on my observation skills would be useful. A failure to meet a test
    case is not always as clear-cut as it may seem. The context of the failure
    and use case may give further insight to the underlying issue. Improving my
    observation skills would allow me to better identify these details when
    manually testing our system.
    \item David: I will focus on observation skills so I can verify that an
    error has been resolved by reproducing the steps exactly. This way I can
    more reliably know when the program is behaving as expected. It will also
    help me look closely at the code to predict why an error may have arisen.
    \item Albert: I will look to improve my observation skills as there are too
    many times that a test fails because I wasn't paying enough attention and
    put the wrong initial conditions.
    \item Gary: I will try to improve my analytical skills because as the
    complexity of our software increases, we have to separate the system into
    different individual parts for testing and then integrate it later.
\end{itemize}

\subsubsection{Learning a New Automation Framework}
Learning how to use new automation frameworks, specifically Cypress, could be
challenging for group members who do not have experience with it.\\

\noindent \textbf{Learning Approaches}
\begin{enumerate}
    \item Dedicate a time during our meetings for Almen, who has professional
    experience using Cypress, to explain how to use Cypress and ensure that
    people have the general knowledge to automate UI test cases.
    \item Learn individually through online documentation, videos, and tutorials
    on Cypress's website.
\end{enumerate}

\begin{itemize}
    \item Almen: Already proficient in developing UI test cases, I would
    gravitate towards the later option to strengthen my understanding on the
    matter and aid my team whenever it arises.
    \item Jonathan: As I do not have any experience with automated testing
    frameworks, I would prefer to dedicate meetings with Almen. This would allow
    me to more efficiently learn how the framework works and specifically how it
    integrates with our specific system and use cases.
    \item Kabishan: My automated testing experience lies mostly in iOS and
    Android testing through Appium. While the foundations of creating test cases
    are the same, irrespective of platform, scheduling some time with Almen will
    allow me to expedite the exploration of any nuances and subtleties of the
    Cypress framework.
    \item David: I will aim to self-learn the Cypress framework, and hopefully
    Almen can provide advice for any specific questions I may have.
    \item Albert: As someone with experience, it would be best to follow Almen
    as she already knows the scope of the framework for our purposes.
    \item Gary: I would prefer to have meeting with Almen and learning about
    this framework as this is more efficient. Learning from documentation can be
    time-consuming and not all the details are needed for the purpose of testing
    this project.
\end{itemize}

\subsubsection{Validating Plans}
It is just as important to know how to verify and validate our test plans as it
is to create the test plans. It can be difficult to set the criteria that
effectively and sufficiently validate the plans. \\

\noindent \textbf{Learning Approaches}
\begin{enumerate}
    \item Learn individually from books or online resources about how
    professionals in the industry verify and validate their plans effectively.
    \item Collaborate with professors and peers to see how their thought process
    when laying out their plans and how they verify and validate them.
\end{enumerate}

\begin{itemize}
    \item Jonathan: I believe that this is an aspect that I have a lot of room
    to improve in. Because of that, I think a mix of both learning approaches
    would suit me best. In my opinion, collaborating with professors and peers
    would be easier to learn from. However, I think that it is also important to
    learn how industry professionals go about verifying and validating their
    plans. 
    \item David: I will collaborate with peers and classmates to see how they
    are verifying their plans. A variety of perspectives is beneficial in
    learning the best way to do something, and I will be able to combine the
    best practices from each person I ask.
    \item Kabishan: In order to better verify and validate our test plan, I will
    learn from books and online resources, and I will also ask professionals in
    the industry about how they perform this task. As enticing as the second
    option sounds, it would be very difficult for me to arrange a time with my
    professors and my peers to master this skill.
    \item Albert: I wish to collaborate with others as it is a more hands on
    experience and can provide more help than what was asked for.
    \item Gary: I personally believe it is always better to learn from
    professional in the industry about how they verify their products and plans
    effectively because they have done it countless number of times and they
    have very mature methodology.
    \item Almen: I would like to approach the latter option of collaborating
    with professors and peers as this provides first hand insight on what is
    needed to verify and validate plans. The former option might have a lot of
    unnecessary content that may not be relevant to us.
\end{itemize}

\subsubsection{Structuring Peer Review}
A well-structured peer review allows us to derive deeper insight on topics that
are important, rather than surface-level feedback on the topic of the reviewer's
choice. It's important to know how to write good questionnaires, for example,
while directing our reviewers' attention to the most impactful topics in our
documentation and design. \\

\noindent \textbf{Learning Approaches}
\begin{enumerate}
    \item Learn from the surveying principles taught in SFWRENG 4HC3, including
    questionnaire and interview formatting
    \item Study peer review formats from previous courses and online resources
\end{enumerate}

\begin{itemize}
    \item David: I will apply the practices we learn in SFWRENG 4HC3 in
    structuring peer review. We have practice with these from assessments in
    that course, and it will be interesting to apply my learning to other
    courses.
    \item Kabishan: In order to ask better questions, and therefore derive more
    effective feedback from our users, I will learn from the surveying
    principles taught in SFWRENG 4HC3. For example, instead of asking open-ended
    questions, which are difficult to compare for multiple participants, asking
    with a range of answers would be better.
    \item Jonathan: I would prefer to apply the principles and knowledge we have
    learned about reviewing from SFWRENG 4HC3. I have found many of the
    principles we have covered in that course interesting and applicable to our
    project.
    \item Albert: There's a lot of research done for understanding people and
    asking the correct questions. Using these resources may be overwhelming, but
    can provide more information than from one course.
    \item Gary: I am happy to apply the knowledge I learned from 4HC3 class and
    I will look through the resources provided from that course to understand
    these principles better.
    \item Almen: I wish to apply my knowledge taught in SFWRENG 4HC3 as the
    course has provided a lot of valuable information in constructing proper
    questionnaires that has been previously assessed in the course.
\end{itemize}

\subsubsection{Static Testing}
Static testing can be very difficult and time consuming for those not used to
it. It requires a full understanding of the program and is not only about
getting the right output. \\

\noindent \textbf{Learning Approaches}
\begin{enumerate}
    \item Practice with small programs on our own and compare against static
    testing tools
    \item Work together on static testing a part of the system and let the main
    contributors lead the testing.
\end{enumerate}

\begin{itemize}
    \item Albert: I have never used static testing tools before, so I think
    comparing my abilities to them will provide a good indication of my
    abilities.
    \item Kabishan: During SFWRENG 3S03, I worked on several practice problems
    that required me to use static testing to identify faulty code that may
    never be uncovered at run-time. In addition to this knowledge, I will
    practice with small programs on my own and compare my results against static
    testing tools, such as SonarQube, to ensure that our project does not crash
    during execution.
    \item David: I will practice on small programs of my own, such as previous
    projects. This will be a good way to see where my shortcomings are, and it
    will be good practice because I won't have worked on these projects for a
    significant time.
    \item Jonathan: I think that practicing on smaller programs would be a more
    effective way to learn about static testing and improve my skills in that
    area. I would then be able to apply the skills I have learned to our project
    when putting our modules through static testing.
    \item Gary: I have learned some static testing tools before but I haven't
    used it for a while, so I think the best way for me to refresh the memory is
    to practice with a small dummy program and refresh the memory.
    \item Almen: I personally would choose to practice on a smaller program,
    such as ones that can be found online, and do a code review on the program.
    I would then compare it with static testing tools to validate what I have
    found. That why, I would be able to improve my skills in static testing and
    be able to contribute to static testing of the project.
\end{itemize}

\subsubsection{API Testing}
Having a back-end service that is intended to handle user web request also needs
to be tested through an API testing tool, such as Postman. API testing can speed
up the testing process compared with UI testing by simply changing the input
parameters of each web request sent to the server and provides a form of early
validation for the data and response of the back-end system.\\

\noindent \textbf{Learning Approaches}
\begin{enumerate}
    \item Download Postman software and follow their user guidelines and check
    their documentation on the website.
    \item Ask a group member who has relevant experience with it.
\end{enumerate}

\begin{itemize}
    \item Jonathan: As I have no experience with the framework, I think that
    downloading the Postman software and going through their guidelines and
    documentation would be the best way for me to learn about the API testing
    framework. I would probably supplement this with online videos or tutorials
    to further understand how the framework works.
    \item Kabishan: Even though I have previously tested endpoints using another
    software, ReadyAPI, I believe that I can still benefit from following the
    user guide provided by Postman, as there may be industry proven techniques
    that identify certain oversights when designing the back-end API. 
    \item David: I plan to download the software and follow their guidelines and
    documentation myself. Postman's website hosts tutorials which I hope I will
    find useful. Any questions I have can be directed to a group member with
    experience to speed up the learning process.
    \item Albert: Learning a new tool can be an interesting experience but can
    also take up a lot of time. Getting tests wrong because you were using the
    tool wrong can lead to even more time loss. Relying on someone else who is
    experienced with it would be best.
    \item Almen: I believe that having organizing a knowledge transfer session
    with a group member with experience in this should be done as they would
    provide a lot more relevant information that will be useful to the project.
    Checking the documentation can be overwhelming and not all the information
    will be used.
    \item Gary: I have prior experience with API testing and Postman therefore I
    can check the documentation when questions comes up.
\end{itemize}

\end{document}