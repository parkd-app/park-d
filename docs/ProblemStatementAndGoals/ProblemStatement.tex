\documentclass[12pt,letterpaper]{article}
\usepackage[utf8]{inputenc}
\usepackage[margin=1in]{geometry}
\usepackage[titletoc,title]{appendix}
\usepackage{graphicx}
\usepackage{booktabs}
\usepackage{hyperref}
\usepackage{tabularx}
\usepackage{indentfirst}
\usepackage{soul}

\title{Problem Statement and Goals\\\progname}

\author{\authname}

\date{\today}

\input{../Comments}
\input{../Common}

\begin{document}

\maketitle

\newpage

\begin{table}[hp]
\caption{Revision History} \label{TblRevisionHistory}
\begin{tabularx}{\textwidth}{lXX}
\toprule
\textbf{Date} & \textbf{Developer(s)} & \textbf{Change}\\
\midrule
Sep 22, 2022 & Albert, Almen, David, Gary, Jonathan, Kabishan & Revision 0\\
\midrule
\color{red}
Apr 4, 2023 & \textcolor{red}{Albert, Almen, David, Gary, Jonathan, Kabishan} &
\textcolor{red}{Revision 1}\\
\bottomrule
\color{black}
\end{tabularx}
\end{table}

\newpage
\tableofcontents
\newpage
\listoftables
\newpage

\section{Problem Statement}

In a car centered city where you drive everywhere, finding a place to park can
either be straightforward or a complete chore. How often do you enter a parking
lot and immediately find a spot to your liking? One that's close to the store,
or handicapped reserved. Do you find yourself more often driving up and down
lanes looking left and right for a spot? While following others doing the same.
What if there was a way that saved you the trouble and showed you an open spot?

\subsection{Problem}
\label{problem}
Drivers waste their valuable time driving around parking lots looking for an
open spot. We aim to drastically reduce this time by finding open spots and
directing drivers to them.

\subsection{Inputs and Outputs}
Included in this section are high level inputs and outputs to characterize the
\nameref{problem}.

\subsubsection{Inputs}
\begin{itemize}
    \item Parking lot cameras or sensors
    \item Parking lot location
    \item Parking spot preference (\textcolor{red}{standard, }handicapped,
    reserved)
    \item User spot selection
\end{itemize}

\subsubsection{Outputs}
\begin{itemize}
    \item Location of open spots
    \item Directions to the selected spots
\end{itemize}

\subsection{Stakeholders}
For our problem, two stakeholders were identified, drivers and parking lot
owner/manager.

\subsubsection{Drivers}
Any person who needs to park their car at a parking lot would be a key
stakeholder for our project as they would be the main user of our product.
Drivers are the main demographic that faces the problem of finding an empty spot
in a parking lot. They would be able to use our product to find empty parking
spots, which would be especially useful in a busy lot.

\subsubsection{Parking Lot Owner/Manager}
Parking lot owners/managers need to give permission to setup cameras or sensors
on their property. They also don't want customers to waste their time in parking
lots when they can be shopping. With fewer drivers wandering around, collisions
and arguments are less likely to occur.

\subsection{Environment}
The system is used in a vehicle by a driver or passenger, so it must comply with
laws relating to device usage by drivers.

\newpage

\section{Goals}

%%%%%%%%%%%%%%%%%%%%%%%%%%%%%% Jonathan %%%%%%%%%%%%%%%%%%%%%%%%%%%%%%
\begin{table}[hp]
\begin{tabularx}{\textwidth}{|X|X|X|}
\toprule
\textbf{Goals} & \textbf{Description} & \textbf{Importance}\\
\midrule
Open Spot Detection & Analyze a video feed of a parking lot and detect whether
parking spots are empty or occupied. & Detecting empty and occupied parking
spots is the core functionality of our program. This goal must be achieved and
the feature must be robust before we can develop other features and aspects of
the project.\\
\hline
Video Translation & Accurately translate locations from a video feed to physical
locations in the parking lot and creating a map of the parking lot. & Detecting
where the parking spots are from a video feed and mapping it out is essential as
we want to provide a map the users can easily understand. They can use that map
to see which spot is most convenient for them as well as easily navigate to
available spots.\\
\hline
Spot Guidance & Guide drivers to the shortest path to empty parking spots. &
Just finding an open parking spot for the user would not be effective as they
would still have to drive around the parking lot looking for the available spot.
Navigating users to the available spot would save them more time as they would
immediately know where to park and how to get there.\\
\bottomrule
\end{tabularx}
\caption{Goals} \label{TblGoals} 
\end{table}

\begin{table}[hp]
\begin{tabularx}{\textwidth}{|X|X|X|}
\toprule
\textbf{Goals} & \textbf{Description} & \textbf{Importance}\\
\midrule
Preference & Drivers can set a preference to specific types of parking spots. &
The ability to set a preference for special parking spots such as Accessible
spots would be important as well as convenient for applicable users. The
preferred spots would be highlighted or automatically selected for users to
easily navigate to.\\
\hline
Special Spot Recognition & Recognize special parking spots such as Accessible or
Reserved spots & We do not want to navigate users to empty parking spots that
they do not have the right to use, such as Reserved and Accessible spots. On the
other hand, we want to highlight the special spots for those who can use them
and/or have set a preference for a special parking spot. \\
\hline
\textcolor{red}{\st{Hands free interaction}} & \textcolor{red}{\st{Users can
interact with the system in a hands-free manner and the system will give
information to the user in a way that minimizes distraction.}} &
\textcolor{red}{\st{Since it is illegal for drivers to touch their phone while
driving, the system needs to be usable in other hands-free means. It is
extremely important for users to be safe and not distracted while using our
system in their vehicles.}}\\
\bottomrule
\end{tabularx}
\caption{Goals continued} \label{TblGoals} 
\end{table}


\newpage

%%%%%%%%%%%%%%%%%%%%%%%%%%%%%% David %%%%%%%%%%%%%%%%%%%%%%%%%%%%%%
\section{Stretch Goals}
%---------EXPAND DESCRIPTIONS/IMPORTANCE------
\begin{table}[hp]
\begin{tabularx}{\textwidth}{|X|X|X|}
\toprule
\textbf{Goals} & \textbf{Description} & \textbf{Importance}\\
\midrule
Immediate Spots & Prioritize giving drivers spots closest to their location. &
Parking lots can be large with confusing layouts and spots should be prioritized
by ease of access by vehicle and by foot. \\
\hline
Vandalism analysis & Appraise the risk of vandalism to a vehicle by analyzing
body language, and detect when it is taking place. & Parking lots are publicly
accessible and vehicles are always at risk of break-ins and damage. This feature
may alleviate the worries that some drivers have regarding their vehicles. \\
\hline
Spacious Spots & Detection of spots with adequate space for a given vehicle,
depending on vehicle dimensions and gaps left adjacent to the space. &
Recommending an unusable space, for example due to poor parking by adjacent
drivers, is a frustrating experience. Our application should minimize the amount
of erroneous parking recommendations.\\
\hline
Reservations & Reserving a parking spot before arriving at the destination. &
Reserving a space can save computation time at the moment of arrival by
front-loading the search for an open parking space. This feature will also help
ensure that a parking spot is not taken after it is given to the driver.\\
\bottomrule
\end{tabularx}
\caption{Stretch Goals} \label{TblStretchGoals}
\end{table}

\begin{table}[hp]
\begin{tabularx}{\textwidth}{|X|X|X|}
\toprule
\textbf{Goals} & \textbf{Description} & \textbf{Importance}\\
\midrule
Hazard Detection & Identification of parking hazards such as damaged pavement,
large puddles, or loose debris. & Hazard detection will help alert drivers to
obstructions in their parking space to minimize erroneous recommendations, in
conjunction with the adequate space detection feature.\\
\hline
Improper Parking Detection & Detection of parking across multiple lines, in
reserved areas such as fire lanes, or across driveways, among other
possibilities. & This feature could be used by parking lot administrators to
administer fines or other warnings. It will help maintain an orderly parking
area, which could be a further incentive to potential partners and users.\\
\hline
Occupancy Tracking & Track the duration of occupancy for a given space and
compare it to an estimation of the average parking duration of the area.
Furthermore, analyze groups which appear to be preparing to leave the parking
lot. & In the event that a parking lot becomes full, providing an estimate of
when a spot may become available can be helpful to the end user. The common
chore of following a exiting group back to their vehicle could be automated. \\
\hline
\textcolor{red}{Hands free interaction} & \textcolor{red}{Users can interact
with the system in a hands-free manner and the system will give information to
the user in a way that minimizes distraction.} & \textcolor{red}{Since it is
illegal for drivers to touch their phone while driving, the system needs to be
usable in other hands-free means. It is extremely important for users to be safe
and not distracted while using our system in their vehicles.}\\
\bottomrule
\end{tabularx}
\caption{Stretch Goals continued} \label{TblStretchGoals2}
\end{table}
\end{document}