\documentclass{article}

\usepackage{tabularx}
\usepackage{booktabs}

\title{Reflection Report on \progname}

\author{\authname}

\date{}

\input{../Comments}
\input{../Common}

\begin{document}

\begin{table}[hp]
\caption{Revision History} \label{TblRevisionHistory}
\begin{tabularx}{\textwidth}{llX}
\toprule
\textbf{Date} & \textbf{Developer(s)} & \textbf{Change}\\
\midrule
April 5, 2023 & Albert, Almen, David, Gary, Jonathan, Kabishan & Revision 1\\
\bottomrule
\end{tabularx}
\end{table}


\newpage

\maketitle

In our busy car-centric cities, finding a place to park can either be
straightforward or a complete chore. How often do you actually enter a parking
lot and immediately find a spot to your liking? Or do you often find yourself
driving up and down, looking left and right for a good spot? What if there was a
way to save you the trouble and show you an open spot directly?

Introducing Park’d, an application that aims to provide a solution to a
bothersome daily chore by clearly showing open spaces to drivers, and providing
directions to reach them. Park’d allows users to view every available parking
space at participating parking lots in real time, while providing live
navigation directions to the space of their choice.

\section{Project Overview}
The objective of Park'd is to leverage machine vision to determine the occupancy
status of spots in parking lots, with minimal additional hardware required. All
that is required for the model to function is a live stream of an overhead
camera view of the parking lot, and annotations for parking spot locations in
the camera view. The requirements for this project were focused on key user
interactions, as well as the relevant distracted driving laws for the province
of Ontario. Users needed to be able to select from all participating parking
lots, select available parking spots, and navigate to them with location access.
Administrators needed to be able to create new parking lots and mark spot
locations on the camera view and the user-facing map view.

\section{Key Accomplishments}

\plt{What went well?  This can be what went well with the documentation, the
  coding, the project management, etc.}
  \begin{itemize}
      \item The project is able to detect parking spots based on real time video
      feed and the result is quite accurate.
      \item The entire project is hosted online which means everyone is able to
      access it.
      \item We consistently upheld our quality standards for delivering
      documentation and code, which resulted in a splendid outcome.
      \item Team members all had well-defined roles and areas of expertise. This
      made delegating tasks much simpler.
      \item Presentations were well-structured, engaging, and went exactly as
      planned.
      \item We followed human computer interface design principles while
      designing our application which makes our product look very professional,
      aesthetically pleasing, as well as promote usability of the project to our
      end users.
      \item We all challenged ourselves by taking on tasks in areas we were
      unfamiliar with which promoted personal development in those areas 
  \end{itemize}

\section{Key Problem Areas}

\plt{What went wrong?  This can be what went wrong with the documentation, the
  technology, the coding, time management, etc.}
  \begin{itemize}
      \item The machine learning model is hardware-demanding which could've been
      more optimized if we retrain the fully-connected layers in the YOLOV3
      architecture
      \item We migrated our project from React to vanilla JavaScript after the
      proof of concept demonstration because React introduced unnecessary
      complication for features that we were not making use of. This led to us
      needing to re-implement some functionality.
      \item Organizing realistic deadlines for six people and holding members
      accountable for them.
      \item With all members having a full course load, it was rather hard to
      manage our time for this course with many deliverables and other courses. 
      \item There were many technologies and tools that could have helped us
      with managing our time/tasks and making some of our tasks easier, such as
      online time management tools with paid plans and paid assets. We decided
      against spending using paid technologies and tools which added more to our
      plates and made tracking tasks a little disorganized
  \end{itemize}

\section{What Would you Do Differently Next Time}
\begin{itemize}
    \item More meetings should be held regarding the development to keep
    everyone in the right track, and internal deadlines should be set to ensure
    work is done in time.
    \item Introducing redundancy into our system by adding more inputs besides
    video footage. For example, using ultrasonic sensor information. We can
    accomplish this by using a MVC pattern, where we can add a model for
    different input types.
    \item Be more reserved in our planning stages and expect features to be
    complicated than anticipated.
    \item Determine what functionality is shared between different parts of the
    application and implement it in one file so that important changes are
    reflected everywhere at once.
    \item Have an even more formalized meeting minutes and being more consistent
    with tracking our issues. It was rather difficult at times to keep track of
    what one was responsible of.
\end{itemize}

\end{document}